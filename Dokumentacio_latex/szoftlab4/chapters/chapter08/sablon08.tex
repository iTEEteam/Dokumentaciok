% Szglab4
% ===========================================================================
%
\chapter{Részletes tervek}

\thispagestyle{fancy}

\section{Osztályok és metódusok tervei}

\subsection{Bullet}
\begin{itemize}
\item Felelősség\\
A tornyok egy Bullet-et tárolnak, amit minden lövésnél átadnak a lövő függvénynek. Ennek a lövedéknek a feladata, hogy az ellenségnek megmondja mennyit sebez rajta.
%\comment{Mi az osztály felelőssége. Kb 1 bekezdés.}
\item Ősosztályok\\
Object
%\comment{Mely osztályokból származik (öröklési hierarchia)\newline
%Legősebb osztály $\rightarrow$ Ősosztály2 $\rightarrow$ Ősosztály3...}
\item Interfészek\\
Nincs
%\comment{Mely interfészeket valósítja meg.}
\item Attribútumok\\
%\comment{Milyen attribútumai vannak}
	\begin{description}
		\item[damage] Az alapsebzés értéke. Ezt adják vissza a getDamage függvények, ha nincs fejlesztve az adott típusú Enemy-re a torony. -, int
		\item[enemyType] A torony itt tárolja, hogy melyik ellenség típusra erősebb a sebzése. -, String
	\end{description}
\item Metódusok\\
%\comment{Milyen publikus metódusokkal rendelkezik. Metódusonként egy-három mondat arról, hogy a metódus mit csinál.}
	\begin{description}
		\item[Bullet(int damage, Enemy enemyType)] Konstruktor
\item[int getHobbitDamage()] Ha hobbitot sebez, ezzel a függvénnyel kérdezi le a sebzés értékét. + 
\item[int getHumanDamage()] Ha embert sebez, ezzel a függvénnyel kérdezi le a sebzés értékét. +
\item[int getDwarfDamage()] Ha törpöt sebez, ezzel a függvénnyel kérdezi le a sebzés értékét. +
\item[int getElfDamage()] Ha tündét sebez, ezzel a függvénnyel kérdezi le a sebzés értékét. +
\item[void setEnemy(String e)] Beállítja az ellenséget akire specializált (enemyType). +
\item[void setDamage(int damage)] Beállítja a lövedék sebzését (damage). +

	\end{description}
\end{itemize}


\subsection{Enemy}
\begin{itemize}
\item Felelősség\\
Az Enemy osztály felelőssége, hogy egy adott lövedék (Bullet) hatására sebződjön, vagy ha már sokat sebződött, akkor haljon meg, valamint az, hogy celláról cellára mozgassa magát, és ha eléri a végzet hegyét értesítse a Game osztályt. Tudja, hogy milyen sebességgel haladt eredetileg, és milyen sebességgel halad most. Ez egy absztrakt ősosztály, ami összefogja a 4 ellenségtípust (Hobbit, Elf, Dwarf, Human).
\item Ősosztályok\\
Object
\item Interfészek\\
IPathPlaceable
\item Attribútumok\\
	\begin{description}
		\item[static final speed] A két lépés között eltelt idő. -, int
		\item[static final maxHP] A maximális életerő értéke. -, int
\item[modSpeed] Az ellenség belső idő mérője. A setModSpeed változtathatja – jellemzően negatív irányba, akadályokon. -, int
\item[nextPath] A soron következő path címe. -, Path
\item[health] Életerejét tárolja ebben. Hurt függvényben csökkenti. \#, int
\item[myPath] Az a mező, ahol tartózkodik. \#, Path
\item[igame] Ezen keresztül tudja módosítani a manát, amikor meghal, illetve ha elér a végzet hegyére módosítani a számlálót (Game.succeededE), hogy nőjön egyel. \#, IGame

	\end{description}
\item Metódusok\\
	\begin{description}
		\item[hurt(Bullet b)] Sebződést megvalósító metódus, abstract, minden Enemy-típusban máshogy implementálódik. +
\item[move()] mozog, a következő path-ra lép, cellát vált. +
\item[Enemy(int sp, int msp, int h, ig: IGame)] konstruktor. +
\item[void setModSpeed(int msp)] A modSpeed változót változtatja. Lassítani lehet vele. +
\item[setHealt(int h)] A health változó settere. +
\item[void cut()] A kettévágásos lövésért felelős metódus. Létrehoz egy új, azonos típusú ellenséget (abstract), feleakkora életerővel. Beteszi az igam enemiesIn tömbjébe. Saját életerjét is felezi. +

	\end{description}
\end{itemize}

\subsection{Enemy subclasses: Elf, Hobbit, Dwarf, Human}
\begin{itemize}
\item Felelősség\\
Sebződés: egy Bullet alapján a saját életét csökkenteni, és ha kell, meghalni. Tehát felüldefiniálja az Enemy ősosztály hurt metódusát.
\item Ősosztályok\\
Object $\rightarrow$ Enemy
\item Interfészek\\
IPathPlaceable

\item Metódusok\\
	\begin{description}
		\item[hurt(Bullet)] Sebződik, a kapott Bullet alapján getDamage... metódus által visszaadott értékkel csökkenti az életerejét. Ha elfogyott az életereje törli magát a pathról és az igame.enemiesIn-ből Ha 0-t kap, cut-ot hív. +
		
	\end{description}
\end{itemize}



\subsection{Game}
\begin{itemize}
\item Felelősség\\
A Game osztály felelőssége többek közt a játék ütemezése, az idő múlásának kontrollálása. Ezenkívül az inicializálás, vagyis a játék kezdeti állapotának felvétele, továbbá a modell állapotának folyamatos változása miatti frissítés, valamint ennek a grafikus felületen való megjelenítése. A game osztály hozza létre az ellenségeket és indítja el őket az úton.
\item Ősosztályok\\
Object
\item Interfészek\\
IGame
\item Attribútumok\\
	\begin{description}
		\item[Map map] játék térképe
\item[enemiesOut] A pályára még be nem lépett ellenségek. Ezt töltjük meg újabb ellenség hullám generálásakor, majd fokozatosan kiürítjük és egyesével átírjuk a tartalmat az enemiesIn-be. -, List<Enemy>
\item[enemiesIn] A pályára már belépett ellenségek. Minden update hívásnál módosul(hat) a tartalom, pl. új Enemy lép a pályára vagy meghal egy-pár. -, List<Enemy>
\item[towers] A tornyok, amik a pályán vannak. Update-enként végigmegyünk rajtuk és meghívjuk a shoot. -, List<Tower>
\item[mana] A maradék varázserő. -, int
\item[firstP] Az út kezdő cellája. Ennek segítségével indítjuk el az Enemy-ket a pályán, megadva nekik a címet. -, Path
\item[noEnemies] Kezdeti hullámérték, amely folyamatosan nő, azt mutatja meg, hogy következő körben hány ellenséget kell létrehozni és beküldeni a pályára. Ebből megtudhatjuk azt is, hogyha a játékos teljesítette a pályát. -, int
\item[succeededE] A végzet hegyét elért enemy-k száma. Ha egy Enemy célba ér, az IGame-n keresztül tudja növelni. Ha elér egy maximum értéket, vége a játéknak. -, int
\item[hazeTime] Az az idő, ameddig a köd tart. Ha lejár a köd, nullázzuk, ameddig a ködnek tartania kell, növeljük. -, int



		
	\end{description}
\item Metódusok\\
	\begin{description}
		\item[void update()] Frissíti a modellt, grafikát. X időközönként folyamatosan hívjuk. Belül az enemiesIn-en végighaladva minden elemen move-ot hívunk, towers-en ugyanígy shoot-ot, illetve ha kell, enemiesOut-ból betesszük a következő elemet. Random módon a towers elemein haze-t is hív. +
\item[void initialize(String name)] Kiinduló állapot felvétele. +
\item[Game()] Konstruktor. +
\item[void makeEnemies()] Létrehoz noEnemies db ellenséget, random típusút, ezeket beteszi az enemiesOut-ba. +

		
		
	\end{description}
\end{itemize}

\subsection{Controller}
\begin{itemize}
\item Felelősség\\
A felhasználó és a játék közötti kommunikációnak véghezvitele a felelőssége. Kristályok, tornyok, akadályok vásárlását lehet rajta keresztül megcsinálni. Ellenőriznie kell, hogy van-e a játékosnak elég varázsereje a tranzakcióhoz.
\item Ősosztályok\\
Object
\item Interfészek\\
Nincs
\item Attribútumok\\
	\begin{description}
		\item[igame] Szükséges függvények elérésére szolgál, hogy tudja módosítani a Game-et. -, int
		\item[chosenField] Ha Field-et választ ki, ebben a változóban tárolja. -, Field
		\item[chosenPath] Ha Path-t választ ki, ebben a változóban tárolja. -, Path
		\item[chosenEnemy] A kiválasztott EnemyType. -, String

		
	\end{description}
\item Metódusok\\
	\begin{description}
		
		\item[void buyTower()] Torony vételére szolgál. Létrehoz egy tornyot, beregisztrálja a chosenField-re. +
		\item[void buyObstacle()] Lásd buyTower Obstacle-el és chosenPath-el. +
		\item[void buySpeed/Range/Damage/EnemyType/Intensiyty/RepairGem()] Kristályok vásárlása. A Towert/Obstacle-t amire vettük a chosenField/Path-en érjük el. +
		\item[void setField(Field f)] A chosenField-et állítja. +
		\item[void setPath(Path p)] A chosenPath-t állítja. +
		\item[void setEnemy(String e)] A chosenEnemy-t állítja. +
		
		
		
	\end{description}
\end{itemize}

\subsection{IGame}
\begin{itemize}
\item Felelősség\\
Az IGame interfész szolgáltatást nyújt az akadályoknak, tornyoknak, ellenségeknek, hogy rajta keresztül manát írjanak jóvá/csökkentsenek, illetve ellenségek esetén a végzet hegyét elért ellenségek számát módosítsák. Speciális interfész a Game osztályhoz. 

\item Metódusok\\
	\begin{description}
		\item[void changeMana()] A manát megváltoztató metódus. +
\item[void incSucceeded()] A succeededE értékét megváltoztató metódus. +
\item[void addTower(Tower t)] A hozzáad egy tornyot a towers listához. +
\item[void removeEnemyIn(Enemy e)] Az ellenség törlése enemiesIn-ből. +
\item[void removeEnemyOut(Enemy e)] Az ellenség törlése enemiesOut-ból. +
\item[void removeTower(Tower t)] A torony törlése towers-ből. +
\item[void addEnemyIn(Enemy e)] Ellenség hozzáadása enemiesIn-hez. +
\item[int getMana()] A mana lekérdezése. +


		
	\end{description}
\end{itemize}

\subsection{Gem}
\begin{itemize}
\item Felelősség\\
A Gem osztály felel a torony fejlesztéséért. Ha a játékos vesz a toronyra/akadályra valamilyen kristályt, akkor jön létre, megkapja a torony, és beépíti magába. 
\item Ősosztályok\\
Object
\item Interfészek\\
Nincs
\item Attribútumok\\
	\begin{description}
		\item[static price] Az ár, amennyi varázserőbe kerül. -, int
	\end{description}
\item Metódusok\\
	\begin{description}
		
		\item[Gem(int price)] Konstruktor. +
		
		
	\end{description}
\end{itemize}

\subsection{IObstacle}
\begin{itemize}
\item Felelősség\\
Olyan metódusok használatát teszi lehetővé, amelyek az Obstacle típusú elemek viselkedését modellezik
\item Ősosztályok\\
Nincs

\item Metódusok\\
	\begin{description}
		\item[void slow(int intesity, Path p)] Szól p-nek, hogy lassítsa le az ellenséget intensity-vel. +
\item[void amortization()] Amortizál, csökkenti amort értékét. +
\item[void increaseIntesity(int intens)] Megnöveli az intesity-t intens-el. +
\item[void addIOGem(IOGem: iog)] Az iog kristályt hozzáadja az akadályhoz. +
\item[repair()] Megjavítja az akadályt, amort értékét maximalizálja. +


		
		
	\end{description}
\end{itemize}
\subsection{Obstacle}
\begin{itemize}
\item Felelősség\\
Az Obstacle osztály felelőssége egyrészt az, hogy amikor áthalad rajta egy ellenség (Enemy) lelassítsa. Másfelől felelőssége az is, hogy egy-egy ellenség  áthaladtával amortizálódjon, valamint ha már teljesen elhasználódott, értesítse azt az út elemet (Path), amelyiken áll.
\item Ősosztályok\\
Nincs
\item Interfészek\\
IObsacle, IPathPlaceable
\item Attribútumok\\
	\begin{description}
	\item[slowIntens] A lassítás mértéke. -, int
\item[myPath] A path, amin rajta van. -, Path
\item[amort] Az elhasználódottság mértéke. -, int
\item[static final price] Az ára. +, int
\item[gems] A megvett kristályok listája. -, ArrayList<IOGem> 


	\end{description}
\item Metódusok\\
	\begin{description}
		
		\item[Obstacle(int intens, Path p, int amort, int price, bool up)] Konstruktor. +
		
		
	\end{description}
\end{itemize}
\subsection{ITower}
\begin{itemize}
\item Felelősség\\
A torony funkciói vannak benne. Egy olyan interfész, amin keresztül a tornyok kezelhető.
\item Metódusok\\
	\begin{description}
		\item[void setPaths()] A myField-ből kiindulva a range-el lefedett területen felkeresi, és beregisztrálja a paths listába a path cellákat. +
\item[void shoot()] A torony akkor lő, ha letelt az újratöltési idő, ekkor megnézi, hogy lőtávon belül van-e ellenség, és ha van meghívja a hurt függvényét, átadva paraméterként a bullet-et. +
\item[void addITGem(ITGem gem)] Paraméterként megkapja a kiválasztott kristályt, bulletet frissíti, ha kell, illetve a torony adott attrimútumait. +
\item[Enemy chooseEnemy()] A torony tárolja a hatókörbe eső path cellákat. Minden tick-ben végig megy rajtuk, és kiválaszt egyet, amelyiken van ellenség, és oda fog lőni. Az ellenséggel tér vissza. +
\item[void haze()] Ködöt generál a tornyon mod range beállításával. +
\item[void clearUp()] Beállítja a modRange-et normál, köd nélküli értékre range alapján. +
\item[+getEnemyType(): String] lekérdezi és visszaadja a Bulletben tárolt enemyType-ot

		
		
	\end{description}
\end{itemize}
\subsection{Tower}
\begin{itemize}
\item Felelősség\\
Az egyetlen tervezett toronytípus, le lehet rakni a pályán az úton kívül bárhova. A hatósugarába belépett ellenségekre lőnie kell, lehet fejleszteni lövési sebességét, erejét, újratöltési idejét és egy ellenségtípusra még hatásosabbá tenni a lövedékeit. A játékos varázserőért tud lerakni tornyokat, illetve el is adhatja őket. Ez a legfontosabb eszköz amivel a játékos meg tudja akadályozni az ellenségek célba jutását.
\item Ősosztályok\\
Nincs
\item Interfészek\\
ITower, IFieldPlaceable
\item Attribútumok\\
	\begin{description}
		\item[static final price] Az ára varázserőben. +, int
\item[range] lőtáv, hatókör. -, int
\item[modRange] Módosított hatókör. -, int
\item[speedCtr] A torony belső idő mérője. Ezt vizsgálja minden lövés előtt, hogy eltelt e elég idő. -, int
\item[speed] két lövés között eltelt minimális idő. -, int
\item[bullet] A torony tárol egy lövedéket, mindig ezt lövi ki. -, Bullet
\item[ArrayList<ITGem> gems] A megvásárolt kristályokat tárolja. -, ArrayList<ITGem> 
\item[myField] A mezőt tárolja amin áll. -, Field 
\item[paths] Hatósugárba eső út cellák. -, ArrayList<Path> 
\item[myField] Mező, amin áll. -, Field 
\item[igame] Egy interfész a játék logikára, amivel a bejutott ellenségek számát, és a varázserőt is lehet állítani. -, IGame 

	\end{description}
\item Metódusok\\
	\begin{description}
		\item[Tower(int rang, int pr)] Konstruktor. +
\item[void upgradeSpeed(int sp)] Fejleszti a lövési sebességét. +
\item[void upgradeRange(int rng)] Fejleszti a lőtávot. +
\item[void upgradeEnemy(Enemy e)] Egy ellenségtípusra növeli a sebzést. +
\item[void upgradeDamage(int dmg)] Növeli a sebzést. +

		
		
	\end{description}
\end{itemize}
\subsection{Map}
\begin{itemize}
\item Felelősség\\
A Map osztály a játéktér elemeit, mint cellák tárolja, egy két dimenziós tömbben. Megadja minden egyes cellához, a szomszédjai referenciáját. 
Nincs
\item Interfészek\\
Nincs
\item Attribútumok\\
	\begin{description}
		\item[name] A pálya neve, egyben az azonosítója. -, String 
		\item[level] A pálya szintje. -, int 
		\item[grid] A cellákat tartalmazó 2 dimenziós tömb. -, Array<Array<Cell>> 

		
	\end{description}
\item Metódusok\\
	\begin{description}
		
		\item[Map(Map map)] Az osztály konstruktora. +
\item[void load(string name)] Megnyitja a paraméterként kapott nevű fájlt, és abból betölti a pálya celláinak tulajdonságait, felépíti a pályát. +
\item[int getLevel()] Visszaadja a pálya szintjét. +
\item[String getName()] Visszaadja a pálya nevét. +
\item[Path getFirstPath()] Visszaadja a pálya belépési pontjának referenciáját. +

		
		
	\end{description}
\end{itemize}
\subsection{Cell}
\begin{itemize}
\item Felelősség\\
A Cell a pálya egy egységét reprezentáló osztály. Létrehozásakor megkapja a 4 szomszédja referenciáját. Maga a cella nem tudja, hogy hol van a térképen. A cella tárolja a rajta éppen tartózkodó ellenségek referenciáit. A Cell osztály absztrakt.
\item Ősosztályok\\
Nincs
\item Interfészek\\
Nincs
\item Attribútumok\\
	\begin{description}
		\item[neighbours] 4 elemű tömb, tárolja 4 irányban a szomszédjai referenciáját. -, Array<Cell> 

		
	\end{description}
\item Metódusok\\
	\begin{description}
		
		\item[Cell(Array<Cell> cells)] Konstruktor, paraméterként kapja a szomszédos mezők referenciáit. +
		\item[bool isPath()] Olyan értékkel tér vissza amilyen típusú a cella (true/false megfelelés). +
		
		
	\end{description}
\end{itemize}
\subsection{Field}
\begin{itemize}
\item Felelősség\\
A Field osztály a Cell osztály leszérmazottja. A nem út típusú cellákat (mező) reprezentálja. Egy mezőre egy torony helyezhető.
\item Ősosztályok\\
Object $\rightarrow$ Cell
\item Interfészek\\
Nincs
\item Attribútumok\\
	\begin{description}
		\item[itower] A mezőn álló torony interfészű elem tárolása. -, ITower 
		\item[igame] IGame referencia, hogy tudja módosítani a Game-et. -, IGame

		
	\end{description}
\item Metódusok\\
	\begin{description}
		
		
		\item[void registerIFieldPlaceable(IFieldPlaceable ifp)] Egy új tornyot ad hozzá a mezőhöz, ifp-en meghívja a registerField-et önmagával, ami a registerITowert hívja. +
		\item[void deleteIFieldPlaceable(IFieldPlaceable ifield)] Eltávolítja a tornyot a mezőről. +
		\item[void registerITower(itower ITower)] Beteszi a fieldbe a kapott tornyot. +
		\item[bool hasTower()] Megadja, hogy szabad-e a mező tower elhelyezéséhez. +
		\item[ITower getITower()] Visszaadja az itowert. +
		\item[Field()] Konstruktor. +
		
		
	\end{description}
\end{itemize}



\subsection{Path}
\begin{itemize}
\item Felelősség\\
A Path a Cell osztály leszármazottja. Az út típusú cellákat reprezentálja. Tartalmazza a rajta lévő ellenségeket és esetleg akadályt. Minden út tudja azt is, hogy hova lehet lépni róla egy lépésben.
\item Ősosztályok\\
Object $\rightarrow$ Cell
\item Interfészek\\
Nincs
\item Attribútumok\\
	\begin{description}
		\item[iobstacle] Az esetleg az úton levő akadályt tárolja. -, IObstacle 
		\item[enemies] Az éppen áthaladó ellenségek listája. -, ArrayList<Enemy> 
		\item[paths] Következő path-ok címei. -, ArrayList<Path> 

		
	\end{description}
\item Metódusok\\
	\begin{description}
		\item[ArrayList<Enemy> hasEnemy()] Visszaadja a rajta lévő ellenségek listáját (enemies-t). +
		\item[bool isPath()] Igaz értékkel tér vissza. +
		\item[void deleteIPathPlaceable(IPathPlaceable ipath)] Kitörli a tárolójából a paraméterként kapott referenciával megegyező tárolt referenciát. +
		\item[void deleteEnemy(Enemy e)] Kitörli enemies-ből a kapott e-t. +
		\item[void registerIPathPlaceable(IPathPlaceable ipath)] Beregisztrálja a paraméterként kapott objektumot, mint saját magán tartózkodó ellenség vagy akadály. RegisterPath-t hív az ipath-on, ami a rá jellemző registert hívja vissza. +
		\item[bool hasEnemy()] Megmutatja, hogy van-e a cellán ellenség. +
		\item[void registerEnemy(Enemy e)] A kapott ellenséget beteszi az enemies-be. +
		\item[void registerObstacel(Obstacle o)] A kapott akadály lesz az obstacle. +
		\item[ArrayList<Enemy> getEnemies()] Visszatér az enemies-el. +
		\item[Path getNext()] Paths-ből ad vissza egy random elemet. +
		\item[bool hasObstacle()] Megadja, hogy van-e rajta Obstacle. +
		
		
	\end{description}
\end{itemize}

\subsection{Towerhez tartozó krsitályok: Range, Speed, Damage, EnemyType}
\begin{itemize}
\item Felelősség\\
A torony lövésének hatósugarát, gyorsítását, sebzését, ellenféltípusra sebzés, növelő kristályok osztályai.
\item Ősosztályok\\
Object $\rightarrow$ Gem
\item Interfészek\\
ITGem
\item Attribútumok\\
	\begin{description}
		\item[range/ speed/ damage/ eType] A kristálynak megfelelő attribútum, amin fejleszt. -, az első 3 int, eType String
		
	\end{description}
\item Metódusok\\
	\begin{description}
		\item[Konstruktorok]
		
	\end{description}
\end{itemize}
\subsection{Obstaclehez tartozó kristályok: Intensity, Repair}
\begin{itemize}
\item Felelősség\\
Akadályra helyezhető kristályok, ami növeli a lassítás értékét vagy megjavítja az akadályt.
\item Ősosztályok\\
Object $\rightarrow$ Gem
\item Interfészek\\
IOGem

\item Attribútumok\\
	\begin{description}
		\item[intensity] Az intensityGem-hez tartozik, intensity értékét állítja az obstacle-ben. A repairGem-nek nincs attribútuma. -, int
		
		
	\end{description}
\item Metódusok\\
	\begin{description}
		\item[Konstruktorok]
		
	\end{description}
\end{itemize}
\subsection{IOGem}
\begin{itemize}
\item Felelősség\\
Akadályra helyezhető kristályok interfésze.

\item Metódusok\\
	\begin{description}
		\item[void upgradeObstacle(Obstacle o)] A kapott akadályt fejleszti. Meghívja a kristályra jellemző, obstacle-ben lévő fejlesztő függvényt. +
		
	\end{description}
\end{itemize}
\subsection{ITGem}
\begin{itemize}
\item Felelősség\\
Toronyra illeszthető kristályok interfésze.

\item Metódusok\\
	\begin{description}
		\item[void upgradeTower(Tower t)] Fejleszti a kapott t tornyot magával. Meghívja a rá jellepző upgrade...  tower metódust. +
\item[int getValue()] Visszaad egy, a torony árával képzett értéket, a torony eladásakor jóváírandó mana érték kiszámításához. +

		
	\end{description}
\end{itemize}
\subsection{IFieldPlaceable}
\begin{itemize}
\item Felelősség\\
Interfész a mezőre helyezhető osztályok számára. Azonosítja azokat az objektumokat, amelyeket csak a mező típusú pályaelem tartalmazhat.

\item Metódusok\\
	\begin{description}
		\item[void registerField(Field field)] A mezőre helyezhető objektumnak megadja  paraméterben annak a mezőnek a referenciáját, amelyikre helyezve lesz. +
\item[void sell()] A mezőre helyezhető objektum eladása, annak megfelelő manát ad a játékosnak amennyit az objektum ér, majd a mező törli magáról az objektumot. +

		
	\end{description}
\end{itemize}
\subsection{IPathPlaceable}
\begin{itemize}
\item Felelősség\\
Interfész az útra helyezhető osztályok számára. Azonosítja azokat az objektumokat, amelyeket csak az út típusú pályaelem tartalmazhat.

\item Metódusok\\
	\begin{description}
		\item[void eliminate(Path p)] Az útra helyezhető objektum eltávolítása az útról (p-ről). +
\item[void registerPath(Path p)] Az útra helyezhető objektumnak megadja paraméterben annak az útnak a referenciáját, amelyikre helyezve lesz. +

	\end{description}
\end{itemize}

\lstset{escapeinside=`', xleftmargin=10pt, frame=single, basicstyle=\ttfamily\footnotesize, language=sh}


\section{A tesztek részletes tervei, leírásuk a teszt nyelvén}
%[A tesztek részletes tervei alatt meg kell adni azokat a bemeneti adatsorozatokat, amelyekkel a program működése ellenőrizhető. Minden bemenő adatsorozathoz definiálni kell, hogy az adatsorozat végrehajtásától a program mely részeinek, funkcióinak ellenőrzését várjuk és konkrétan milyen eredményekre számítunk, ezek az eredmények hogyan vethetők össze a bemenetekkel.]

\subsection{Ellenség mozgás és sebzés}
\begin{itemize}
\item Leírás\newline
A tesztelő létrehoz 2 ellenséget és vesz 2 tornyot, egyiket a 3 kristállyal, a másikat Elf kristállyal. Ez után utukra indítja az ellenségeket, akik útközben meghalnak. Végén elad a játékos egy tornyot.
\item Ellenőrzött funkcionalitás, várható hibahelyek \newline
Ellenségek létrehozása, halála. Tornyok vétele, fejlesztése, tüzelés és eladást tesztel. 

Várható hibahely: ellenség mozgatása, kristály vétel
\item Tesztesethez használt térkép \newline
\begin{lstlisting}
testmap1.txt

x	x	x	x	x	x
en	r	r	r	r	ex
x	x	x	x	x	x
\end{lstlisting}

\item Bemenet\newline
\begin{lstlisting}
loadmap testmap1.txt
buy tower 2-1
buy tower 2-4
buy gem 2-1 speed
buy gem 2-1 range
buy gem 2-1 damage
buy gem 2-4 enemy elf
make hobbit 1-1
update 2
make elf 1-1
update 2
sell 2-1
exit
\end{lstlisting}
\item Elvárt kimenet\newline
\begin{lstlisting}
map testmap1.txt loaded
Tower0 created
Tower1 created
SpeedGem0 created
RangeGem0 created
DamageGem0 created
EnemyTypeGem0 created
Hobbit0 created
update
Hobbit0 moves
Tower0 shoots Hobbit0
Hobbit0 damage 40
update
Hobbit0 moves
Tower0 shoots Hobbit0
Hobbit0 damage 40
Tower1 shoots Hobbit0
Hobbit0 damage 20
Hobbit0 died
Elf0 created
update
Elf0 moves
Tower0 shoots Elf0
Elf0 damage 40
update
Elf0 moves
Tower0 shoots Elf0
Elf0 damage 40
Tower1 shoots Elf0
Elf0 damage 60
Elf0 died
Tower0 sold for 80 mana
VEGE
\end{lstlisting}
\end{itemize}

\subsection{Akadály teszt}
\begin{itemize}
\item Leírás\newline
A pályán elhelyez a játékos egy akadályt, amit fejleszt egy intenzitás növelő kristállyal, majd bejut az ellenség a végzet hegyéhez. Amikor az utolsó ellenség is áthaladt rajta elhasználódik.
\item Ellenőrzött funkcionalitás, várható hibahelyek\newline
Ellenségek létrehozása, akadály létrehozása, fejlesztése, elhasználódása. Végzet hegyére jutás.

Várható hibahely: akadály elhasználódása, ellenség bejutása a Végzet Hegyéhez
\item Tesztesethez használt térkép \newline
\begin{lstlisting}
testmap2.txt:

x	x	x	x
en	r	r	ex
x	x	x	x
\end{lstlisting}
\item Bemenet\newline
\begin{lstlisting}
loadmap testmap2.txt
buy obstacle 1-1
buy gem 1-1 intensity
make hobbit 1-0
update 14
exit
\end{lstlisting}
\item Elvárt kimenet\newline
\begin{lstlisting}
map testmap2.txt loaded
Obstacle0 created
IntensityGem0 created
Hobbit0 created
update
Hobbit0 moves
Obstacle0 slows Hobbit0 by 10
Obstacle0 worn out
update
update
update
update
update
update
update
update
update
update
update
Hobbit0 moves
update
Hobbit0 moves
update
Hobbit0 moves
Hobbit0 reached Mount Doom
VEGE
\end{lstlisting}
\end{itemize}

\subsection{Köd, és ellenség kettészelés}
\begin{itemize}
\item Leírás\newline
A pályára feltesz a tesztelő egy Human ellenséget és egy tornyot. Frissíti a játékot, a torony lő az ellenségre. Beállítja a ködöt, és hogy kettészelő lövedékeket lőjenek a tornyok. Kétszer frissíti a pályát, elsőre csak lép, mert kikerült a torony hatósugarából, másodiknak újra bekerült, kettészelő lövedékkel lövi meg a torony.	
\item Ellenőrzött funkcionalitás, várható hibahelyek\newline
Köd, kettélövő lövedék.

Várható hibahely: hatósugár újrakalibrálása, ellenség kettészelése
\item Tesztesethez használt térkép \newline
\begin{lstlisting}
testmap3.txt:

x	x	x	x	x
en	r	r	r	ex
x	x	x	x	x
\end{lstlisting}
\item Bemenet\newline
\begin{lstlisting}
loadmap testmap3.txt
make human 1-0
buy tower 2-4
buy gem 2-4 range
update 1
haze on
cut on
update 2
exit
\end{lstlisting}
\item Elvárt kimenet\newline
\begin{lstlisting}
map testmap3.txt loaded
Human0 created
Tower0 created
RangeGem0 created
update
Human0 moves
Tower0 shoots Human0
Human0 damage 20
Haze is turned on
Cut is turned on
update
Human0 moves
update
Human0 moves
Tower0 shoots Human0
Human0 damage 0
Human0 cut in half
Human1 created
VEGE
\end{lstlisting}
\end{itemize}

\subsection{Ellenség elágazáshoz ér}
\begin{itemize}
\item Leírás\newline
A pályára feltesz a tesztelő 3 különböző ellenséget. A pálya kialakításából adódóan 1 lépés után elágazáshoz érnek. Mindegyiket különböző irányba küldjük. 
\item Ellenőrzött funkcionalitás, várható hibahelyek\newline
Ellenségek elágazás kezelése.

Várható hibahely: elágazás
\item Tesztesethez használt térkép \newline
\begin{lstlisting}
testmap4.txt:

x	x	x	x	x	x
x	x	r	r	d	x
en	r	rud	r	r	ex
x	x	r	r	u	x
x	x	x	x	x	x
\end{lstlisting}
\item Bemenet\newline
\begin{lstlisting}
loadmap testmap4.txt
make elf 2-1
make human 2-1
make dwarf 2-1
update 1
route Elf0 1
route Human0 0
route Dwarf0 2
update 1
drawMap
exit
\end{lstlisting}
\item Elvárt kimenet\newline
\begin{lstlisting}
map testmap4.txt loaded
Elf0 created
Human0 created
Dwarf0 created
update
Elf0 moves
Human0 moves
Dwarf0 moves
update
Elf0 moves
Elf0 crossroad
Human0 moves
Human0 crossroad
Dwarf0 moves
Dwarf0 crossroad
/-----------------\
|xx|xx|xx|xx|xx|xx|
|xx|xx|xx|xx|xx|xx|
|--+--+--+--+--+--|
|xx|xx|e |  |  |xx|
|xx|xx|  |  |  |xx|
|--+--+--+--+--+--|
|  |  |  |e |  |  |
|  |  |  |  |  |  |
|--+--+--+--+--+--|
|xx|xx|e |  |  |xx|
|xx|xx|  |  |  |xx|
|--+--+--+--+--+--|
|xx|xx|xx|xx|xx|xx|
|xx|xx|xx|xx|xx|xx|
\-----------------/
VEGE
\end{lstlisting}
\end{itemize}

\section{A tesztelést támogató programok tervei}
%\comment{A tesztadatok előállítására, a tesztek eredményeinek kiértékelésére szolgáló segédprogramok részletes terveit kell elkészíteni.}

Készülnek a prototípus program, illetve a szövegfájl összehasonlító segédprogram fordítását és futtatását megkönnyítő batch fájlok, melyek futtatásuk után automatizáltan elvégzik ezeket a feladatokat.

Az előre elkészített, és a 8.2-es pontban leírt tesztesetek amellett, hogy futtathatóak a parancssorból a bemenet átirányításával a megfelelő fájlból (pl.: java ProtoTester < teszt1.dat), reprodukálhatjuk őket az erre a célra készített batch fájl segítségével is: test.bat.
A batch fájl működése a következő: futtatás után ellenőrzi, hogy a szükséges programok (prototípus, illetve szöveges fájl összehasonlító) rendelkezésre állnak-e a megfelelő helyen, ha nem, akkor felkéri a felhasználót azoknak a lefordítására, vagy áthelyezésére, majd kilép. Amennyiben rendelkezésre állnak, megkérdezi a tesztelőtől, hogy melyik tesztesetet kívánja lefuttatni, illetve a kimenetet megjelenítse-e a konzolablakban. Amennyiben a megjelenítés mellett dönt a tesztelő, az adott teszteset lefuttatása után keletkező kimenet a konzolban jelenik meg, és billentyűlenyomásig ott is marad, így az elemezhető, az elvárt kimenettel összevethető. Ellenkező esetben, tehát ha a tesztelő úgy dönt, hogy ne jelenjen meg a prototípus kimenete, akkor egy fájlba irányítja azt a batch fájl, és az így keletkező fájlt összeveti az elvárt kimenettel a TxtComparer segédprogram segítségével (ennek specifikációja lentebb található), és megmondja, hogy egyezik-e, majd billentyűlenyomás után kilép.

A tesztelést tovább segíti a TxtComparer nevű, szöveges fájlok összehasonlítását végző program. A program Java nyelven készül, forráskódja a prototípushoz hasonlóan adott. A program a fordítás után a parancssorból futtatható, ahol két paramétert vár: a két összehasonlítandó fájl nevét (pl.: java TxtComparer file1 file2). 
Futtatás után, amennyiben a helyes paraméterek rendelkezésre állnak, illetve elérhetőek a megadott fájlok, a program soronként összehasonlítja a két fájlt, és amennyiben azok megegyeznek, értesíti erről a felhasználót. Azon esetben, amikor a fájlok tartalmában eltérést észlel, az eltérést tartalmazó sorok számai jelennek meg a kimeneten.
