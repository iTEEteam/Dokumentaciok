% Szglab4
% ===========================================================================
%
\chapter{Grafikus felület specifikációja}

\thispagestyle{fancy}

\section{A grafikus interfész}
%\comment{A menürendszer, a kezelői felület grafikus képe. A grafikus felület megjelenését, a használt ikonokat, stb screenshot-szerű képekkel kell bemutatni. Az építészetben ez a homlokzati terv.}

\begin{figure}[H]
\begin{center}
\includegraphics[width=17cm]{chapters/chapter11/images/GraphicPlans.jpg}
\caption{Grafikus interfész}
\label{fig:Grafikus}
\end{center}
\end{figure}

\section{A grafikus rendszer architektúrája}
%\comment{A felület működésének elve, a grafikus rendszer architektúrája (struktúra diagramok). A struktúra diagramokon a prototípus azon és csak azon osztályainak is szerepelnie kell, amelyekhez a grafikus felületet létrehozó osztályok kapcsolódnak.}

\begin{figure}[H]
\begin{center}
\includegraphics[width=17cm]{chapters/chapter11/images/GraphClassesD.jpg}
\caption{Grafikus rendszer struktúra diagram}
\label{fig:Grafikus_struktura}
\end{center}
\end{figure}

\subsection{A felület működési elve}
%\comment{Le kell írni, hogy a grafikai megjelenésért felelős osztályok, objektumok hogyan kapcsolódnak a meglevő rendszerhez, a megjelenítés során mi volt az alapelv. Törekedni kell az MVC megvalósításra. Alapelvek lehetnek: \textbf{push} alapú: a modell értesíti a felületet, hogy változott; \textbf{pull} alapú: a felület kérdezi le a modellt, hogy változott-e; \textbf{kevert}: a kettő kombinációja.}

A grafikus felület kialakításánál elsődleges célunk az volt, hogy a program modelljébe ne épüljön bele a megjelenítés  mivolta, könnyen cserélhető legyen.

Ezért létrehoztunk a megjelenítendő elemeknek egy-egy testvér osztályt amik az objektum kirajzolásáért felelnek.
Két interfésszel általánosítottuk a grafikus objektumokat.

Az IGraphic interfészt megvalósító osztályok a draw() metódussal tudják kirajzolni magukat.

Az IView interfészen keresztül tudnak a testvérosztályaikra hivatkozni, és a notify() metóduson keresztül értesítheti, hogy változott, aminek hatására kirajzolja magát.

Ez a megoldás, miszerint modell beli objektum minden változáskor értesíti a felületet push alapú megoldás.



\subsection{A felület osztály-struktúrája}
%\comment{Osztálydiagram. Minden új osztály, és azon régiek, akik az újakhoz közvetlenül kapcsolódnak.}

\begin{figure}[H]
\begin{center}
\includegraphics[width=17cm]{chapters/chapter11/images/SimaClassDiag.jpg}
\caption{Osztály diagram}
\label{fig:Osztaly_diagram}
\end{center}
\end{figure}


\section{A grafikus objektumok felsorolása}
%\comment{Az új osztályok felsorolása. Az régi osztályok közül azoknak a felsorolása, ahol változás volt. Ezek esetén csak a változásokat kell leírni.}

\subsection{GPath}
\begin{itemize}
\item Felelősség\\
A GPath felel a Path osztály megjelenítéséért, kirajzolásáért. Minden Path-hoz tartozik egy, és mindegyikhez egy Path tartozik.
\item Ősosztályok\\
Object, Graphic abstract
\item Interfészek\\
IView
\item Attribútumok\\
	\begin{description}
		\item[-path: Path] Referencia arra az útra, amelyikhez tartozik. Ezen keresztül kérheti le a szükséges adatokat a modellből.
		\item[-gEnemies: ArrayList<GEnemy>] Az path enemies-hez tartozó GEnemy-k listája.
\end{description}
\item Metódusok\\
	\begin{description}
		\item[+drawGHuman/GHobbit/GElf/GDwarf: void] A különböző GEnemy-khez tartozó rajzoló függvény.
		\item[+getPath(): Path] Getter path-hoz
		\item[+GPath(p: Path)] Konstruktor.
	\end{description}
\end{itemize}

\subsection{GField}
\begin{itemize}
\item Felelősség\\
A GField felel a Field osztály megjelenítéséért, kirajzolásáért. Minden Field-hez tartozik egy, és mindegyikhez egy Field tartozik.
\item Ősosztályok\\
Object, Graphic abstract
\item Interfészek\\
IView
\item Attribútumok\\
	\begin{description}
		\item[-field: Field] Referencia arra  mezőre, amelyikhez tartozik. Ezen keresztül kérheti le a szükséges adatokat a modellből.
\end{description}
\item Metódusok\\
	\begin{description}
		\item[+getField(): void] Getter fieldhez.
		\item[+GField(f: Field)] Konstruktor.
	\end{description}
\end{itemize}

\subsection{GGame}
\begin{itemize}
\item Felelősség\\
A GGame felel a Game osztály megjelenítéséért. Elsősorban a felső sávért, vagyis a Game azon attribútumainak megjelenéséért, amiket a játékosnak célszerű látnia: mana, bejutott ellenségek száma, pálya szintje stb. 
\item Ősosztályok\\
Object, Graphic abstract
\item Interfészek\\
IView
\item Attribútumok\\
	\begin{description}
		\item[-game: Game] A Game referencia, ami ahhoz kell, hogy meg tudja jeleníteni az adatokat.
		\item[-gEnemies] ArrayList<GEnemy>: A Game enemiesIn és enemiesOut-ban lévő Enemy-khez tartozó megjelenítők listája. 
\end{description}
\item Metódusok\\
	\begin{description}
		\item[+getGame()] Game: Getter game-hez.
		\item[+GGame(g: Game)] Konstruktor.
	\end{description}
\end{itemize}

\subsection{GEnemy és leszármazottai: GHobbit, GHuman, GElf, GDwarf}
\begin{itemize}
\item Felelősség\\
A GEnemy felel azért, hogy a GPath tudja, hogy milyen típusú Enemy-ket kell rajzolni. Ezt az enemy attribútumból tudja meg, illetve drawMe metódus különféle implementációinak segítségével üzen GEnemy a GPath-nak. Minden Enemy-hez létrehozáskor társítunk egy példányt, amit elhelyezünk a GGame-ben.
\item Ősosztályok\\
Object
\item Interfészek\\
Nincs
\item Attribútumok\\
	\begin{description}
		\item[-enemy: Enemy] Ez a hivatkozási pont a modellbeli Enemy-re.
\end{description}
\item Metódusok\\
	\begin{description}
		\item[+drawMe(gp: GPath): void] Ez a metódus gondoskodik arról, hogy a megfelelő típusú Enemy kirajzolása történjen meg a Path-on. Visszahívja gp az enemy típusának megfelelő drawXXX() metódust.
		\item[+GEnemy(e: Enemy)] Konstruktor.
	\end{description}
\end{itemize}

\subsection{Controller}
\begin{itemize}
\item Attribútumok\\
	\begin{description}
		\item[-iview: IView] Interfész az értesítéshez.
\end{description}
\item Metódusok\\
	\begin{description}
		\item[+getIView()] IView: Getter az iview-hoz.
	\end{description}
\end{itemize}

\subsection{GController}
\begin{itemize}
\item Felelősség\\
A GController irányítja a grafikus elemeket, végzi a felhasználói interakciók érvényre jutásának a felületre vonatkozó részeit, illetve kezdeményezi a modell megváltoztatását.
\item Ősosztályok\\
Object, Graphic abstract
\item Interfészek\\
IView
\item Attribútumok\\
	\begin{description}
		\item[-control: Controller] Ez a hivatkozási pont a modellhez.
		\item[-chosenGPath: Path] A kijelölt út, pl akadály vétele.
		\item[-chosenGField: Field] A kijelölt mező, pl torony vétele.
		\item[-ggame: GGame] Ez a hivatkozási pont a grafika másik részéhez.
\end{description}
\item Metódusok\\
	\begin{description}
		\item[+set chosenGField/GPath(f/p Field/Path)] void: GField/GPath setter.
		\item[+get chosenGField/GPath(): void] GField/GPath getter.
		\item[+buyGSpeed/Range/Damage/EnemyType/Intesity/RepairGem(): void] Gem-ek vétele.
		\item[+buyTower/Obstacle()] Tower/Obstacle vétele.
	\end{description}
\end{itemize}

\subsection{IView}
\begin{itemize}
\item Felelősség\\
Az IView interfész teremt kapcsolatot a modellből a grafikus felület felé. Segítségével lehet értesíteni a felületet a modellbeli változásokról. Ehhez különböző értesítő metódusokat tartalmaz.
\item Ősosztályok\\
Nincs
\item Interfészek\\
Nincs
\item Metódusok\\
	\begin{description}
		\item[+notify()] void: Alap értesítő metódus.
		\item[+addGEnemy(ge: GEnemy)] void: Hozzáadunk az iview-hoz egy GEnemy-t.
		\item[+getGEnemy(e: Enemy)] void: Lekérjük az e-hez tartozó GEnemy-t az iview-ból.
		\item[+deleteGEnemy(e: Enemy)] void: Töröljük az e-hez tartozó GEnemy-t az iview-ból.
	\end{description}
\end{itemize}

\subsection{Graphic}
\begin{itemize}
\item Felelősség\\
Az Graphic a rajzoló abstract ősosztály. Különböző rajzolásokat tesz lehetővé.
\item Ősosztályok\\
Nincs
\item Interfészek\\
Nincs
\item Metódusok\\
	\begin{description}
		\item[-draw()] void: Alap rajzoló metódus.
		\item[+highlight()] void: Kijelölő metódus, pl mezőre kattintáshoz.
		\item[+deHighlight()] void: Highlight inverze.
	\end{description}
\end{itemize}

\subsection{Cell}
\begin{itemize}
\item Attribútumok\\
	\begin{description}
		\item[\#iview: IView] Interfész a grafika értesítéséhez.
		\item[\#igame: IGame] A Path és Field igame-je felkerült ide, protected lett.
\end{description}
\item Metódusok\\
	\begin{description}
		\item[+getIView()] IView: Getter az iview-hoz.
	\end{description}
\end{itemize}

\subsection{Game és IGame}
\begin{itemize}
\item Attribútumok\\
	\begin{description}
		\item[-iview: IView] Game-ben interfész az értesítéshez.
\end{description}
\item Metódusok\\
	\begin{description}
		\item[+getIView(): IView] IGame-be getter, hogy mások is elérjék, akik a Gamet elérik az interfészén keresztül.
	\end{description}
\end{itemize}



\section{Kapcsolat az alkalmazói rendszerrel}
%\comment{Szekvencia-diagramokon ábrázolni kell a grafikus rendszer működését. Konzisztens kell legyen az előző alfejezetekkel. Minden metódus, ami ott szerepel, fel kell tűnjön valamelyik szekvenciában. Minden metódusnak, ami szekvenciában szerepel, szereplnie kell a valamelyik osztálydiagramon.}
\subsection{Akadály elhelyezése}
\begin{figure}[H]
\begin{center}
\includegraphics[width=17cm]{chapters/chapter11/images/Akadaly_elhelyezese.jpg}
\caption{Akadály elhelyezése szekvenciadiagram}
\label{fig:Akadaly_elhelyezese}
\end{center}
\end{figure}

\subsection{Akadály javítása}
\begin{figure}[H]
\begin{center}
\includegraphics[width=17cm]{chapters/chapter11/images/Akadaly_javitasa.jpg}
\caption{Akadály javítása szekvenciadiagram}
\label{fig:Akadaly_javitasa}
\end{center}
\end{figure}

\subsection{Akadály törlődése}
\begin{figure}[H]
\begin{center}
\includegraphics[width=17cm]{chapters/chapter11/images/Akadaly_torlodese.jpg}
\caption{Akadály törlődése szekvenciadiagram}
\label{fig:Akadaly_torlodese}
\end{center}
\end{figure}

\subsection{Betöltés}
\begin{figure}[H]
\begin{center}
\includegraphics[width=13cm]{chapters/chapter11/images/Betoltes.jpg}
\caption{Betöltés szekvenciadiagram}
\label{fig:Betoltes}
\end{center}
\end{figure}

\subsection{Ellenség beérkezik}
\begin{figure}[H]
\begin{center}
\includegraphics[width=17cm]{chapters/chapter11/images/Ellenseg_beerkezik.jpg}
\caption{Ellenség beérkezik szekvenciadiagram}
\label{fig:Ellenseg_beerkezik}
\end{center}
\end{figure}

\subsection{Ellenség halála}
\begin{figure}[H]
\begin{center}
\includegraphics[width=17cm]{chapters/chapter11/images/Ellenseg_halala.jpg}
\caption{Ellenség halála szekvenciadiagram}
\label{fig:Ellenseg_halala}
\end{center}
\end{figure}

\subsection{Ellenség lépése}
\begin{figure}[H]
\begin{center}
\includegraphics[width=17cm]{chapters/chapter11/images/Ellenseg_lepese.jpg}
\caption{Ellenség lépése szekvenciadiagram}
\label{fig:Ellenseg_lepese}
\end{center}
\end{figure}

\subsection{Ellenség létrehozása}
\begin{figure}[H]
\begin{center}
\includegraphics[width=17cm]{chapters/chapter11/images/Ellenseg_letrehozasa.jpg}
\caption{Ellenség létrehozása szekvenciadiagram}
\label{fig:Ellenseg_letrehozasa}
\end{center}
\end{figure}

\subsection{Ellenség sebzése}
\begin{figure}[H]
\begin{center}
\includegraphics[width=17cm]{chapters/chapter11/images/Ellenseg_sebzese.jpg}
\caption{Ellenség sebzése szekvenciadiagram}
\label{fig:Ellenseg_sebzese}
\end{center}
\end{figure}

\subsection{Torony elhelyezése}
\begin{figure}[H]
\begin{center}
\includegraphics[width=17cm]{chapters/chapter11/images/Torony_elhelyezese.jpg}
\caption{Torony elhelyezése szekvenciadiagram}
\label{fig:Torony_elhelyezese}
\end{center}
\end{figure}

\subsection{Torony fejlesztése ellenség típusra}
\begin{figure}[H]
\begin{center}
\includegraphics[width=17cm]{chapters/chapter11/images/Torony_fejlesztese_ellenseg_tipusra.jpg}
\caption{Torony fejlesztése ellenség típusra szekvenciadiagram}
\label{fig:Torony_fejlesztese_ellenseg_tipusra}
\end{center}
\end{figure}

\subsection{Torony törlődése}
\begin{figure}[H]
\begin{center}
\includegraphics[width=17cm]{chapters/chapter11/images/Torony_torlodese.jpg}
\caption{Torony törlődése szekvenciadiagram}
\label{fig:Torony_torlodese}
\end{center}
\end{figure}

\subsection{Varázserő változása}
\begin{figure}[H]
\begin{center}
\includegraphics[width=17cm]{chapters/chapter11/images/Varazsero_valtozasa.jpg}
\caption{Varázserő változása szekvenciadiagram}
\label{fig:Varazsero_valtozasa}
\end{center}
\end{figure}










