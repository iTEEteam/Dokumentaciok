% Szglab4
% ===========================================================================
%
\chapter{Szkeleton beadás}

\thispagestyle{fancy}

\section{Fordítási és futtatási útmutató}
%\comment{A feltöltött program fordításával és futtatásával kapcsolatos útmutatás. Ennek tartalmaznia kell leltárszerűen az egyes fájlok pontos nevét, méretét byte-ban, keletkezési idejét, valamint azt, hogy a fájlban mi került megvalósításra.}

\subsection{Fájllista}

\begin{fajllista}

\fajl
{Application.java}
{454}
{2014.03.11.}
{Névvel megegyező osztály.}

\fajl
{Bullet.java}
{1956}
{2014.03.11.}
{Névvel megegyező osztály.}

\fajl
{Cell.java}
{642}
{2014.03.11.}
{Névvel megegyező osztály.}

\fajl
{Controller.java}
{4503}
{2014.03.11.}
{Névvel megegyező osztály.}

\fajl
{DamageGem.java}
{859}
{2014.03.11.}
{Névvel megegyező osztály.}

\fajl
{Dwarf.java}
{589}
{2014.03.11.}
{Névvel megegyező osztály.}

\fajl
{Elf.java}
{576}
{2014.03.11.}
{Névvel megegyező osztály.}

\fajl
{Enemy.java}
{2013}
{2014.03.11.}
{Névvel megegyező osztály.}

\fajl
{EnemyTypeGem.java}
{765}
{2014.03.11.}
{Névvel megegyező osztály.}

\fajl
{Field.java}
{2026}
{2014.03.11.}
{Névvel megegyező osztály.}

\fajl
{Game.java}
{2489}
{2014.03.11.}
{Névvel megegyező osztály.}

\fajl
{Gem.java}
{412}
{2014.03.11.}
{Névvel megegyező osztály.}

\fajl
{Hobbit.java}
{952}
{2014.03.11.}
{Névvel megegyező osztály.}

\fajl
{Human.java}
{587}
{2014.03.11.}
{Névvel megegyező osztály.}

\fajl
{IFieldPlaceable.java}
{300}
{2014.03.11.}
{Névvel megegyező osztály.}

\fajl
{IGame.java}
{486}
{2014.03.11.}
{Névvel megegyező osztály.}

\fajl
{IntensityGem.java}
{774}
{2014.03.11.}
{Névvel megegyező osztály.}

\fajl
{IObstacle.java}
{382}
{2014.03.11.}
{Névvel megegyező osztály.}

\fajl
{IOGem.java}
{260}
{2014.03.11.}
{Névvel megegyező osztály.}

\fajl
{IPathPlaceable.java}
{303}
{2014.03.11.}
{Névvel megegyező osztály.}

\fajl
{ITGem.java}
{278}
{2014.03.11.}
{Névvel megegyező osztály.}

\fajl
{ITower.java}
{328}
{2014.03.11.}
{Névvel megegyező osztály.}

\fajl
{Map.java}
{1108}
{2014.03.11.}
{Névvel megegyező osztály.}

\fajl
{Obstacle.java}
{2398}
{2014.03.11.}
{Névvel megegyező osztály.}

\fajl
{Path.java}
{4024}
{2014.03.11.}
{Névvel megegyező osztály.}

\fajl
{RangeGem.java}
{825}
{2014.03.11.}
{Névvel megegyező osztály.}

\fajl
{RepairGem.java}
{653}
{2014.03.11.}
{Névvel megegyező osztály.}

\fajl
{SkeletonTester.java}
{10032}
{2014.03.11.}
{A szkeleton programot tesztelhetővé tevő program.}

\fajl
{SpeedGem.java}
{711}
{2014.03.11.}
{Névvel megegyező osztály.}

\fajl
{Tower.java}
{3984}
{2014.03.11.}
{Névvel megegyező osztály.}

\fajl
{SkeltonCompiler.bat}
{119}
{2014.03.23.}
{Parancssoros fordítást végzi.}

\fajl
{SkeltonRun.bat}
{97}
{2014.03.23.}
{Futtatja a lefordított szkeletont.}

\end{fajllista}

\subsection{Fordítás}
%\comment{A fenti listában szereplő forrásfájlokból milyen műveletekkel lehet a bináris, futtatható kódot előállítani. Az előállításhoz csak a 2. Követelmények c. dokumentumban leírt környezetet szabad előírni.}

%\%lstset{escapeinside=`', xleftmargin=10pt, frame=single, basicstyle=\ttfamily\footnotesize, language=sh}
%\begin{lstlisting}
%javac -d bin *.java
%\end{lstlisting}

A szkeleton fordítását legegyszerűbben a parancssorból tudjuk elvégezni. A parancssorból közvetlenül elérhetőnek kell lennie a javac.exe fordítóprogramnak. A fordítást követően létrehozásra kerülnek a lefordított .class kiterjesztésű fájlok.

A fordítást egyszerűbb a program mellé mellékelt SkeletonComplier.bat fájl megnyitásával elvégezni.

\subsection{Futtatás}
%\comment{A futtatható kód elindításával kapcsolatos teendők leírása. Az indításhoz csak a 2. Követelmények c. dokumentumban leírt környezetet szabad előírni.}

%\lstset{escapeinside=`', xleftmargin=10pt, frame=single, basicstyle=\ttfamily\footnotesize, language=sh}
%\begin{lstlisting}
%cd bin
%java Main.java
%\end{lstlisting}


A szkeleton futtatását a java.exe futtatóprogrammal lehet elvégezni. A parancssorból közvetlenül elérhetőnek kell lennie a java.exe fordítóprogramnak. 

A futtatást egyszerűbb a program mellé mellékelt SkeletonRun.bat fájl megnyitásával elvégezni.

\section{Értékelés}
%\comment{A projekt kezdete óta az értékelésig eltelt időben tagokra bontva, százalékban.}
\newenvironment{ertekelesplusz}
{
	\begin{center}
	\begin{longtable}{| p{3cm} | r | r | p{6cm} |}
	\hline
	\multicolumn{1}{|p{3cm}|}{\textbf{Tag}} &
	\multicolumn{1}{l|}{\textbf{Munkaóra}}&
	\multicolumn{1}{l|}{\textbf{Munka százalékban}} &
	\multicolumn{1}{r|}{\textbf{Aláírás}} \tabularnewline \hline \hline 
	\endfirsthead

	\hline
	\multicolumn{1}{|p{3cm}|}{\textbf{Tag}} &
	\multicolumn{1}{l|}{\textbf{Munkaóra}}&
	\multicolumn{1}{l|}{\textbf{Munka százalékban}} &
	\multicolumn{1}{r|}{\textbf{Aláírás}} \tabularnewline \hline \hline 
	\endhead

}
{
	\end{longtable}
	\end{center}
}

\begin{ertekelesplusz}
\tag{Elekes} % Tag neve
{44,4} %óra
{22}        % Munka szazalekban
\tag{Fuksz}
{36,25}
{22}
\tag{Nagy}
{20,5}
{10}
\tag{Rédey}
{38,75}
{23}
\tag{Seres}
{41,6}
{23}

\end{ertekelesplusz}

