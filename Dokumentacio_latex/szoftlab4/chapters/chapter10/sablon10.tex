% Szglab4
% ===========================================================================
%
\chapter{Prototípus beadása}

\thispagestyle{fancy}

\section{Fordítási és futtatási útmutató}
%\comment{A feltöltött program fordításával és futtatásával kapcsolatos útmutatás. Ennek tartalmaznia kell leltárszerűen az egyes fájlok pontos nevét, méretét byte-ban, keletkezési idejét, valamint azt, hogy a fájlban mi került megvalósításra.}

\subsection{Fájllista}

\begin{fajllista}

\fajl
{proto/Application.java}
{407}
{2014.03.11.}
{Névvel megegyező osztály.}

\fajl
{proto/Bullet.java}
{2176}
{2014.03.11.}
{Névvel megegyező osztály.}

\fajl
{proto/Cell.java}
{677}
{2014.03.11.}
{Névvel megegyező osztály.}

\fajl
{proto/Controller.java}
{4601}
{2014.03.11.}
{Névvel megegyező osztály.}

\fajl
{proto/DamageGem.java}
{459}
{2014.03.11.}
{Névvel megegyező osztály.}

\fajl
{proto/Dwarf.java}
{1325}
{2014.03.11.}
{Névvel megegyező osztály.}

\fajl
{proto/Elf.java}
{1310}
{2014.03.11.}
{Névvel megegyező osztály.}

\fajl
{proto/Enemy.java}
{4436}
{2014.03.11.}
{Névvel megegyező osztály.}

\fajl
{proto/EnemyTypeGem.java}
{504}
{2014.03.11.}
{Névvel megegyező osztály.}

\fajl
{proto/Field.java}
{1222}
{2014.03.11.}
{Névvel megegyező osztály.}

\fajl
{proto/Game.java}
{5368}
{2014.03.11.}
{Névvel megegyező osztály.}

\fajl
{proto/Gem.java}
{274}
{2014.03.11.}
{Névvel megegyező osztály.}

\fajl
{proto/Hobbit.java}
{1330}
{2014.03.11.}
{Névvel megegyező osztály.}

\fajl
{proto/Human.java}
{1323}
{2014.03.11.}
{Névvel megegyező osztály.}

\fajl
{proto/IFieldPlaceable.java}
{127}
{2014.03.11.}
{Névvel megegyező osztály.}

\fajl
{proto/IGame.java}
{470}
{2014.03.11.}
{Névvel megegyező osztály.}

\fajl
{proto/IntensityGem.java}
{460}
{2014.03.11.}
{Névvel megegyező osztály.}

\fajl
{proto/IObstacle.java}
{216}
{2014.03.11.}
{Névvel megegyező osztály.}

\fajl
{proto/IOGem.java}
{96}
{2014.03.11.}
{Névvel megegyező osztály.}

\fajl
{proto/IPathPlaceable.java}
{129}
{2014.03.11.}
{Névvel megegyező osztály.}

\fajl
{proto/ITGem.java}
{115}
{2014.03.11.}
{Névvel megegyező osztály.}

\fajl
{proto/ITower.java}
{235}
{2014.03.11.}
{Névvel megegyező osztály.}

\fajl
{proto/Map.java}
{4763}
{2014.03.11.}
{Névvel megegyező osztály.}

\fajl
{proto/Obstacle.java}
{1815}
{2014.03.11.}
{Névvel megegyező osztály.}

\fajl
{proto/Path.java}
{4517}
{2014.03.11.}
{Névvel megegyező osztály.}

\fajl
{proto/ProtoTester.java}
{17002}
{2014.03.11.}
{A proto programot tesztelhetővé tevő program.}

\fajl
{proto/RangeGem.java}
{420}
{2014.03.11.}
{Névvel megegyező osztály.}

\fajl
{proto/RepairGem.java}
{297}
{2014.03.11.}
{Névvel megegyező osztály.}

\fajl
{proto/SpeedGem.java}
{461}
{2014.03.11.}
{Névvel megegyező osztály.}

\fajl
{proto/Tower.java}
{4416}
{2014.03.11.}
{Névvel megegyező osztály.}

\fajl
{txtComparer/TxtComparer.java}
{1929}
{2014.04.20.}
{A kapott kimenetet az elvárt kimenettel összehasonlító program.}

\fajl
{test1.expected}
{545}
{2014.04.20.}
{Teszteset1 elvárt kimenete}

\fajl
{test2.expected}
{341}
{2014.04.20.}
{Teszteset2 elvárt kimenete}

\fajl
{test3.expected}
{300}
{2014.04.20.}
{Teszteset3 elvárt kimenete}

\fajl
{test4.expected}
{561}
{2014.04.20.}
{Teszteset4 elvárt kimenete}

\fajl
{test1.txt}
{200}
{2014.04.09.}
{Teszteset1 bemenete}

\fajl
{test2.txt}
{97}
{2014.04.09.}
{Teszteset2 bemenete}

\fajl
{test3.txt}
{115}
{2014.04.09.}
{Teszteset3 bemenete}

\fajl
{test4.txt}
{147}
{2014.04.09.}
{Teszteset4 bemenete}

\fajl
{testmap1.txt}
{39}
{2014.04.07.}
{Teszteset1 térképe}

\fajl
{testmap2.txt}
{27}
{2014.04.07.}
{Teszteset2 térképe}

\fajl
{testmap3.txt}
{33}
{2014.04.07.}
{Teszteset3 térképe}

\fajl
{testmap4.txt}
{67}
{2014.04.07.}
{Teszteset4 térképe}

\fajl
{ProtoCompileAndRun.bat}
{288}
{2014.04.21.}
{Parancssoros fordítást végzi.}

\fajl
{ProtoTestAndCompare1.bat}
{588}
{2014.04.20.}
{Teszteset1 futtatása és ellenőrzése}

\fajl
{ProtoTestAndCompare2.bat}
{588}
{2014.04.20.}
{Teszteset2 futtatása és ellenőrzése}

\fajl
{ProtoTestAndCompare3.bat}
{588}
{2014.04.20.}
{Teszteset3 futtatása és ellenőrzése}

\fajl
{ProtoTestAndCompare4.bat}
{588}
{2014.04.20.}
{Teszteset4 futtatása és ellenőrzése}

\end{fajllista}

\lstset{escapeinside=`', xleftmargin=10pt, frame=single, basicstyle=\ttfamily\footnotesize, language=sh}

\subsection{Fordítás}
%\comment{A fenti listában szereplő forrásfájlokból milyen műveletekkel lehet a bináris, futtatható kódot előállítani. Az előállításhoz csak a 2. Követelmények c. dokumentumban leírt környezetet szabad előírni.}

A fordítás a ''javac'' programmal történik, ezért olyan környezetben lehetséges csak, ahol ez a program elérhető. 

A tesztelés megkönnyítése végett a fordítás és futtatás automatizálására több batch fájl áll rendelkezésre. Ezeknek a használata a 10.1.3 pontban van részletezve. 

A fordítás parancssorból is lehetséges a következő, a project gyökerében kiadott paranccsal:
\begin{lstlisting}
javac -d ./bin/ proto/*.java
\end{lstlisting}
Ez a parancs lefordítja a prototípust, és a lefordított állományokat a bin mappába helyezi.  

A szövegfájlok összehasonlítására használt segédprogram (TxtComparer) fordításához a projekt gyökerében adjuk ki a következő parancsot:
\begin{lstlisting}
javac -d ./bin/ txtComparer/TxtComparer.java
\end{lstlisting}

\subsection{Futtatás}
%\comment{A futtatható kód elindításával kapcsolatos teendők leírása. Az indításhoz csak a 2. Követelmények c. dokumentumban leírt környezetet szabad előírni.}

A futtatás a ''java'' programmal történik, ezért olyan környezetben lehetséges csak, ahol ez a program elérhető. 

Az adott batch fájlok segítségével lehetséges lefordítani és futtatni a prototípust: ehhez a ProtoCompileAndRun.bat-ot futtassuk, ami a fordítás után azonnal el is indítja a prototípust.

Ha a tesztelést az előre megírt teszteseteken keresztül végezzük, akkor a batch fájlokkal ez is automatizálva van: ProtoTestAndCompare(szám).bat fájlok, amelyek a prototípust lefordítják, az adott teszt bemenetével lefuttatják, és a kimenetet egy fájlba irányítják, amit aztán a szövegfájlok összehasonlítására használt segédprogram (TxtComparer) segítségével összehasonlítanak az elvárt bemenetet tartalmazó fájllal, majd végül az eredményt kiírják.

Ha a futtatást parancssorból akarjuk végezni, lehetőség van arra is. Amennyiben a 10.1.2 pontban leírt módon végeztük a fordítást, a futtatás a következő, a project gyökerében kiadott paranccsal végezhető:

\begin{lstlisting}
java -classpath .\bin\ proto.ProtoTester
\end{lstlisting}
Ez a parancs elindítja a prototípust.
Ha egy adott tesztesetet akarunk lefuttatni, akkor azt a bemenet átirányításával tehetjük meg:
\begin{lstlisting}
java -classpath .\bin\ proto.ProtoTester < test1.txt
\end{lstlisting}
Ekkor a test1.txt-ben megadott parancsok hajtódnak végre a prototípusban.
Ha az adott teszteset kimenetét össze akarjuk vetni az elvárt kimenettel, akkor futtassuk a prototípust a be- és kimenet átirányításával, majd az így kapott kimeneti fájlt és az elvárt kimenetet tartalmazó fájlt megadva paraméterként, futtassuk a szövegfájlok összehasonlítására használt segédprogramot (TxtComparer):
\begin{lstlisting}
java -classpath .\bin\ proto.ProtoTester < test2.txt > test2.out
java -classpath .\bin\ txtComparer.TxtComparer test2.out test2.expected
\end{lstlisting}

A szövegfájlok összehasonlítására használt segédprogram futtatásához, amennyiben a 10.1.2 pontban leírt módon végeztük a fordítást, a projekt gyökerében adjuk ki a következő parancsot:
\begin{lstlisting}
java -classpath .\bin\ txtComparer.TxtComparer (egyik fajl) (masik fajl)
\end{lstlisting}

\section{Tesztek jegyzőkönyvei}

\subsection{Teszteset1}
%\comment{Az alábbi táblázatot az utolsó, sikeres tesztfuttatáshoz kell kitölteni}
%\comment{Az alábbi táblázatot a megismételt (hibás) tesztek esetén kell kitölteni minden ismétléshez egyszer. Ha szükséges, akkor a valós kimenet is mellékelhető mint a teszt eredménye.}

\tesztok
{Elekes Tamás}
{2014.04.21.}

\tesztfail
{Elekes Tamás}
{2014.04.19}
{Az enemyTypeGem-mel fejlesztett torony nem növelte a sebzést.}
{A kristályt nem regisztrálta be, elírás.}
{A bullet-ben az ellenségtípusok nagy betűvel kezdődtek, míg a kristályban kis betűvel voltak.}
\subsection{Teszteset2}

\tesztok
{Rédey Bálint, Elekes Tamás}
{2014.04.21}

\tesztfail
{Elekes Tamás}
{2014.04.20}
{Az obstacle nem használódott el.}
{Rossz értékre van állítva az obstacle életereje, vagy nem csökkenti.}
{Az akadály életerejét akkorára állítottam hogy 1 ellenség is tönkretegye.}

\subsection{Teszteset3}

\tesztok
{Elekes Tamás}
{2014.04.21.}

\tesztfail
{Rédey Bálint, Elekes Tamás}
{2014.04.21}
{Hibás: első update után nem sebzi meg human0-t tower0, második update után kettévágja, harmadik után szintén.}
{tower.setPaths nem működik helyesen, torony rossz ütemben lő}
{Bemeneti parancs, kimenet rosszul volt megadva, azon változtattunk.}

\tesztfail
{Elekes Tamás}
{2014.04.19}
{A haze mintha nem csökkentené a hatósugarat.}
{A haze() metódus beállította az új range értéket, majd az upgradeRange() metódust hívta. Ami újra megváltoztatta a range-t.}
{A haze upgradeRange helyett a setPaths()-t hívja.}

\subsection{Teszteset4}

\tesztok
{Rédey Bálint}
{2014.04.21. 15:45}

\tesztfail
{Rédey Bálint}
{2014.04.21}
{Hibás: nem írja a kereszteződéshez érést: crossroad, kirajzoláskor rossz helyen vannak az enemy-k}
{Hiányzik valahol egy kiírás, drawmap vagy az ellenségek mozgása nem működik helyesen.}
{Enemy.move-ban elhelyeztem a crossroad kiírását, ami hiányzott, ehhez Path-ba elhelyeztem egy getNextPaths() metódust, ami a nextPaths listával tér vissza. Minden más helyesen működött, a megadott parancsok közt volt hiba: kereszteződésnél a dwarf0 visszafele lépett.}

\section{Értékelés}
%\comment{A projekt kezdete óta az értékelésig eltelt időben tagokra bontva, százalékban.}


\begin{ertekelesplusz}
\tag{Elekes} % Tag neve
{67,1} %óra
{25}        % Munka szazalekban
\tag{Fuksz}
{59,2}
{24}
\tag{Nagy}
{31}
{5}
\tag{Rédey}
{59,2}
{24}
\tag{Seres}
{59,3}
{22}

\end{ertekelesplusz}
