% Szglab4
% ===========================================================================
%
\chapter{Analízis modell kidolgozása 1}

\thispagestyle{fancy}

\section{Objektum katalógus}

%\comment{Minden, a feladatban szereplő objektum rövid, egy-két bekezdés hosszú ismertetése. Meg kell jelenjen minden objektumhoz, hogy mi a felelőssége. Informális leírás, ezért nem kell foglalkozni az örökléssel, az interfészekkel, az absztrakt osztályokkal, a segédosztályokkal.}

%\subsection{Objektum1}
%\comment{Felelősség informális leírása}
%
%\subsection{Objektum2}
%\comment{Felelősség informális leírása}



\subsection{Akadály (Obstacle)}
Az akadály (Obstacle) objektum felelőssége egyrészt az, hogy amikor áthalad rajta egy ellenség (Enemy) lelassítsa. Másfelől felelőssége az is, hogy egy-egy ellenség  áthaladtával amortizálódjon, valamint ha már teljesen elhasználódott, értesítse azt az út elemet (Path), amelyiken áll.
\subsection{Ellenség (Enemy)}
	Egy ellenséget (tünde, hobbit, törp vagy ember) megvalósító objektum. Nyilvántartja a saját életét, sebességét, és azt is, hogy pillanatnyilag melyik mezőn (Field) tartózkodik. Az ő felelőssége még, hogy egy adott lövedék (Bullet) hatására, mennyire sebződjön, vagy ha már sokat sebződött, akkor haljon meg.
\subsection{GemStat}
	Ez az osztály azért felel, hogy egy toronyról (Tower) egyszerűen le tudjuk kérdezni a tulajdonságait. Milyen kristályokat vett már rá a játékos, az egész torony értéke mennyi.
\subsection{Játék (Game)}
A játék (Game) objektum felelőssége többek közt a játék ütemezése, az idő múlásának kontrollálása. Ezenkívül az inicializálás, vagyis a játék kezdeti állapotának felvétele, továbbá a modell állapotának folyamatos változása miatti frissítés, valamint ennek a grafikus felületen való megjelenítése.

\subsection{Kristály (Gem)}
	Ha a játékos vesz a toronyra/akadályra valamilyen kristályt, akkor jön létre, megkapja a torony, és beépíti magába. Felelőssége, hogy általa érvényre jussanak a fejlesztések. 
\subsection{Lövedék (Bullet)}
	A tornyok egy lövedéket tárolnak, amit minden lövésnél átadnak a lövő függvénynek. Ennek a lövedéknek a feladata, hogy az ellenségnek megmondja mennyit sebez rajta.
\subsection{Mező (Field)}
A Field osztály a Cell osztály leszármazottja. A nem út típusú cellákat (mező) reprezentálja. Egy mezőre egy torony helyezhető. 

\subsection{Pálya (Map)}
A Map osztály a játéktér elemeit, mint cellák tárolja, egy két dimenziós tömbben. Megadja minden egyes cellához, a szomszédjai referenciáját. A pályák egy külső XML fájlban kerülnek tárolásra. A pálya ebből a fájlból töltődik be. Az XML fájlban tárolódik a pálya neve, nehézségi szintje, és a térkép struktúrája. Megadja, hogy a tartalmazott cella út vagy mező típusú-e. Minden út cella tartalmaz egy tulajdonságot, ami következő cella irányát adja meg.


\subsection{Torony (Tower)}
	Az egyetlen tervezett toronytípus, le lehet rakni a pályán az úton kívül bárhova. A hatósugarába belépett ellenségekre lőnie kell, lehet fejleszteni lövési sebességét, erejét, újratöltési idejét és egy ellenségtípusra még hatásosabbá tenni a lövedékeit. A játékos varázserőért tud lerakni, illetve eladni tornyokat. Ez a legfontosabb eszköz amivel a játékos meg tudja akadályozni az ellenségek célbajutását.
\subsection{Út (Path)}
A Path a Cell osztály leszármazottja. Az út típusú cellákat reprezentálja. Tartalmazza a rajta lévő ellenségeket és esetleg akadályt.


\section{Statikus struktúra diagramok}
%\comment{Az előző alfejezet osztályainak kapcsolatait és publikus metódusait bemutató osztálydiagram(ok). Tipikus hibalehetőségek: csillag-topológia, szigetek.}

\begin{figure}[h]
\begin{center}
\includegraphics[width=17cm]{chapters/chapter03/images/Main.jpg}
\caption{Osztálydiagram}
\label{fig:Osztálydiagram}
\end{center}
\end{figure}


\section{Osztályok leírása}
%\comment{Az előző alfejezetben tárgyalt objektumok felelősségének formalizálása attribútumokká, metódusokká. Csak publikus metódusok szerepelhetnek. Ebben az alfejezetben megjelennek az interfészek, az öröklés, az absztrakt osztályok. Segédosztályokra még mindig nincs szükség. Az osztályok ABC sorrendben kövessék egymást. Interfészek esetén az Interfészek, Attribútumok pontok kimaradnak.}

\subsection{Bullet}
\begin{itemize}
\item Felelősség\\
Ellenség kapja meg, és ebből tudja meg mennyire sebződik.
%\comment{Mi az osztály felelőssége. Kb 1 bekezdés.}
\item Ősosztályok\\
Object
%\comment{Mely osztályokból származik (öröklési hierarchia)\newline
%Legősebb osztály $\rightarrow$ Ősosztály2 $\rightarrow$ Ősosztály3...}
\item Interfészek\\
Nincs
%\comment{Mely interfészeket valósítja meg.}
\item Attribútumok%
%\comment{Milyen attribútumai vannak}
	\begin{itemize}
		\item int damage: alapsebzés
		\item Enemy enemyType: a torony itt tárolja, hogy melyik ellenség típusra erősebb a sebzése
	\end{itemize}
\item Metódusok%
%\comment{Milyen publikus metódusokkal rendelkezik. Metódusonként egy-három mondat arról, hogy a metódus mit csinál.}
	\begin{itemize}
		\item void refresh(Gem gem): frissíti kristály vásárlás után a sebzési értékeket.
		\item int getHobbitDamage(): ha hobbitot sebez, ezzel a függvénnyel kérdezi le a sebzés értékét. 
		\item int getHumanDamage(): ha embert sebez, ezzel a függvénnyel kérdezi le a sebzés értékét.
		\item int getDwarfDamage(): ha törpöt sebez, ezzel a függvénnyel kérdezi le a sebzés értékét.
		\item int getElfDamage(): ha tündét sebez, ezzel a függvénnyel kérdezi le a sebzés értékét
		\item Bullet(int damage, Enemy enemyType): konstruktor
		
		
	\end{itemize}
\end{itemize}


\subsection{Enemy}
\begin{itemize}
\item Felelősség\\
Tudja, hogy mennyi élete van még, milyen sebességgel haladt eredetileg, és milyen sebességgel halad most. Ez egy absztrakt ősosztály, ami összefogja a 4 ellenségtípust (Hobbit, Elf, Dwarf, Human).
\item Ősosztályok\\
Object
\item Interfészek\\
IPathPlaceable
\item Attribútumok%
	\begin{itemize}
		\item int speed: a normál sebessége
		\item int modSpeed: a módosított sebessége (ha egy akadályon halad keresztül)
		\item health: élete
		\item myPath: az a mező, ahol tartózkodik
	\end{itemize}
\item Metódusok
	\begin{itemize}
		\item Metódusok
		\item registerPath(Path p): új mezőre léptetik
		\item eliminate(): meghal
		\item hurt(Bullet): sebződik (abstract method)
		\item move(): mozog, a következő path-ra lép, cellát vált
		\item Enemy(int sp, int msp, int h): konstruktor
		
		
	\end{itemize}
\end{itemize}

\subsection{Enemy subclasses: Elf, Hobbit, Dwarf, Human}
\begin{itemize}
\item Felelősség\\
Sebződés: egy Bullet alapján a saját életét csökkenteni, és ha kell, meghalni.Tehát felüldefiniálja az Enemy ősosztály hurt metódusát.
\item Ősosztályok\\
Object $\rightarrow$ Enemy
\item Interfészek\\
IPathPlaceable

\item Metódusok
	\begin{itemize}
		\item hurt(Bullet): sebződik
		
	\end{itemize}
\end{itemize}

\subsection{ITower}
\begin{itemize}
\item Felelősség\\
A torony funkciói vannak benne.
\item Metódusok
	\begin{itemize}
		\item void setPaths(): a saját cellájából kiindulva a hatósugarával lefedett területen felkeresi, és beregisztrálja a paths listába a path cellákat.
		\item void shoot(): A torony akkor lő, ha letelt az újratöltési idő, ekkor megnézi, hogy lőtávon belül van-e ellenség, és ha van meghívja a sebzés függvényét, átadva paraméterként a lövedékét. 
		\item void addGem(Gem gem): paraméterként megkapja a kiválasztott kristályt, a gameStat-ot frissíti, és a bullet-et is.
		\item void sell(): A játékos eladhatja a tornyot, amiért a fejlesztések árának felét kapja meg.A torony eltűnik és felszabadul az alatta lévő hely. Az értékét a gemStat változóból 	nyeri ki.
		
		
	\end{itemize}
\end{itemize}

\subsection{Game}
\begin{itemize}
\item Felelősség\\
Lásd objektum katalógus.
\item Ősosztályok\\
Object
\item Interfészek\\
Nincs
\item Attribútumok
	\begin{itemize}
		\item Map map: játék térképe
		\item List<Enemy> enemies: ellenségek listája
		\item static int mana: maradék varázserő


		
	\end{itemize}
\item Metódusok
	\begin{itemize}
		\item void update(): frissíti a modellt, grafikát
		\item void initialize(): kezdeti állapotot felveszi
		\item void tick(): ütemező függvény
		\item static void changeMana(): mana jóváírására, csökkentésére (fejlesztés/eladás bekövetkeztekor)
		\item Game(Map map, ArrayList<Enemy> enemies, int mana): konstruktor
		
		
	\end{itemize}
\end{itemize}
\subsection{GemStat}
\begin{itemize}
\item Felelősség\\
 Lásd objektumkatalógus.
\item Ősosztályok\\
Object
\item Interfészek\\
Nincs
\item Attribútumok
	\begin{itemize}
		\item boolean isDamage: vásároltak-e már sebzésnövelő kristályt a toronyra
		\item boolean isRange: vásároltak-e már hatósugárnövelő kristályt a toronyra
		\item boolean isSpeed: vásároltak-e már lövés sebességnövelő kristályt a toronyra
		\item boolean isEnemy: vásároltak-e már ellenségre specializáló kristályt a toronyra 

		
	\end{itemize}
\item Metódusok
	\begin{itemize}
	
		\item void refresh(Gem gem): egy Kristályt kap, amit beépít a toronyba
		\item int getValue: kiszámolja a torony értékét
		
		
	\end{itemize}
\end{itemize}
\subsection{Gem}
\begin{itemize}
\item Felelősség\\
A kristály osztály frissíti a torony GemStat-ját és Bullet-jét.
\item Ősosztályok\\
Object
\item Interfészek\\
Nincs
\item Attribútumok
	\begin{itemize}
		\item int damage: sebzés növelés értéke
		\item int range: hatósugár növelés értéke
		\item int speed: két lövés közt eltelt idő
		\item static int price: az ára, amennyi manába kerül
		\item Enemy enemyType: ellenség típus amire erősíti a tornyot
	\end{itemize}
\item Metódusok
	\begin{itemize}
		
		\item Gem(int damage, int range, int speed, int price, Enemy enemyType): konstruktor
		
		
	\end{itemize}
\end{itemize}

\subsection{IObstacle}
\begin{itemize}
\item Felelősség\\
Olyan metódusok használatát teszi lehetővé, amelyek az Obstacle típusú elemek viselkedését modellezik.
\item Ősosztályok\\
Nincs

\item Metódusok
	\begin{itemize}
		\item void slow(int intesity, Path p): szól p-nek, hogy lassítsa le az ellenséget intensity-vel
		\item void amortization(): amortizál
		
		
	\end{itemize}
\end{itemize}
\subsection{Obstacle}
\begin{itemize}
\item Felelősség\\
Lásd objektum katalógus. 
\item Ősosztályok\\
Nincs
\item Interfészek\\
IObsacle, IPathPlaceable
\item Attribútumok
	\begin{itemize}
		\item int slowIntens: lassítás mértéke
		\item Path myPath: a mező, amin rajta van
		\item int amort: az elhasználódottság mértéke
		\item static int price: az ára
		\item bool upgrade: volt-e fejlesztve 
		
	\end{itemize}
\item Metódusok
	\begin{itemize}
		
		\item void slow(int intesity, Path p): lelassítja a p-n lévő ellenséget intesity-vel
		\item void amortization(): csökkenti az amort értékét, ha nulla lesz, hívja az eliminate-et
		\item void eliminate(Path p): szól a p-nek, hogy távolítsa el az akadályt
		\item void registerPath(Path p): myPath-t beállítja p-re
		\item void increaseIntensity(int intensity): fejleszztéskor megnöveli a lassytás mértékét
		\item bool isUpgraded(): upgrade-l tér vissza
		\item Obstacle(int intens, Path p, int amort, int price, bool up): konstruktor
		
		
	\end{itemize}
\end{itemize}
\subsection{Tower}
\begin{itemize}
\item Felelősség\\
Lásd objektumkatalógus
\item Ősosztályok\\
Nincs
\item Interfészek\\
ITower, IFieldPlaceable
\item Attribútumok
	\begin{itemize}
		\item Bullet bullet: A torony tárol egy lövedéket, mindig ezt lövi ki.
		\item GemStat gemStat: A megvásárolt kristályokról tárol  statisztikát.
		\item List<Path> paths: Hatósugárba eső út cellák.
		\item Field myField: mező, amin áll
		\item static int price: az ára
		\item int range: lőtáv, hatókör

		
	\end{itemize}
\item Metódusok
	\begin{itemize}
		\item Enemy chooseEnemy(): A torony tárolja a környező path cellákat. Minden tick-ben végig megy rajtuk, és kiválaszt egyet, amelyiken van ellenség, és oda fog lőni. Azzal tér vissza, hogy sikerült-e ellenséget találni.
		\item void register(Field field): beregisztrálja, hogy melyik mezőn van
		\item setPaths(): beállítja a paths-t
		\item shoot(): enemy kiválasztása után rátüzel
		\item sell(): tornyot eladjuk: töröljük a mezőjéről és jóváírjuk az érte kapott összeget
		\item addGem(Gem g): fejlesztéskor a kapott g Gem alapján beállítjuk a range-t és a bullet tagváltozóit
		\item Tower(int rang, int pr): konstruktor
		
		
	\end{itemize}
\end{itemize}
\subsection{Map}
\begin{itemize}
\item Felelősség\\
Ld. objektum katalógus
\item Ősosztályok\\
Nincs
\item Interfészek\\
Nincs
\item Attribútumok
	\begin{itemize}
		\item String name: a pálya neve, egyben az azonosítója
		\item int level: a pálya szintje
		\item Array<Array<Cell>> grid: A cellákat tartalmazó 2 dimenziós tömb

		
	\end{itemize}
\item Metódusok
	\begin{itemize}
		
		\item Map(string name): az osztály konstruktora, a paraméterként megadott névvel rendelkező fájlból betölti a pálya térképét
		\item void load(string name): megnyitja a paraméterként kapott nevű XML fájlt, és abból betölti a pálya celláinak tulajdonságait, felépíti a pályát.
		
		
	\end{itemize}
\end{itemize}
\subsection{Cell}
\begin{itemize}
\item Felelősség\\
A Cell a pálya egy egységét reprezentáló osztály. Létrehozásakor megkapja a 4 szomszédja referenciáját. Maga a cella nem tudja, hogy hol van a térképen. A cella tárolja a rajta éppen tartózkodó ellenségek referenciáit. A Cell osztály absztrakt.
\item Ősosztályok\\
Nincs
\item Interfészek\\
Nincs
\item Attribútumok
	\begin{itemize}
		\item Array<Cell> neighbours: 4 elemű tömb, tárolja 4 irányban a szomszédjai referenciáját.

		
	\end{itemize}
\item Metódusok
	\begin{itemize}
		
		\item Cell(Array<Cell>): konstruktor, paraméterként kapja a szomszédos mezők referenciáit.
		\item bool isPath(): olyan értékkel tér vissza amilyen típusú a cella 
		
		
	\end{itemize}
\end{itemize}
\subsection{Field}
\begin{itemize}
\item Felelősség\\
Ld. objektum katalógus
\item Ősosztályok\\
Object $\rightarrow$ Cell
\item Interfészek\\
Nincs
\item Attribútumok
	\begin{itemize}
		\item ITower itower: a mezőn álló torony interfészű elem tárolása

		
	\end{itemize}
\item Metódusok
	\begin{itemize}
		
		\item bool isPath(): hamis értékkel tér vissza
		\item void addIFieldPlaceable(): egy új tornyot ad hozzá a mezőhöz
		\item void deleteIFieldPlaceable(IFieldPlaceable ifield): eltávolítja a tornyot a mezőről
		\item void registerITower(itower ITower): beteszi ifieldbe a kapott tornyot
		\item Field(): konstruktor
		
		
	\end{itemize}
\end{itemize}



\subsection{Path}
\begin{itemize}
\item Felelősség\\
Ld objektum katalógus
\item Ősosztályok\\
Object $\rightarrow$ Cell
\item Interfészek\\
Nincs
\item Attribútumok
	\begin{itemize}
		\item IObstacle iobstacle: az esetleg az úton levő akadályt tárolja
		\item ArrayList<Enemy> enemies: az éppen áthaladó ellenségek listája
		\item ArrayList<Path> paths: következő path-ok címei

		
	\end{itemize}
\item Metódusok
	\begin{itemize}
		\item ArrayList<Enemy> hasEnemy(): visszaadja a rajta lévő ellenségek listáját
		\item bool isPath(): igaz értékkel tér vissza
		\item void deleteIPathPlaceable(IPathPlaceable ipath): kitörli a tárolójából a paraméterként kapott referenciával megegyező tárolt referenciát
		\item void registerIPathPlaceable(IPathPlaceable ipath): beregisztrálja a paraméterként kapott objektumot, mint saját magán tartózkodó ellenség
		\item bool hasEnemy(): megmutatja, hogy van-e a cellán ellenség
		\item void registerEnemy(Enemy e): a kapott ellenséget beteszi az enemies-be
		\item void registerObstacel(Obstacle o): a o kapott akadály lesz az obstacle
		\item ArrayList<Enemy> getEnemies(): visszatér az enemies-el
		\item Path getNext(): paths-ből ad vissza egy elemet
		
		
	\end{itemize}
\end{itemize}

%\section{Statikus struktúra diagramok}
%\comment{Az előző alfejezet osztályainak kapcsolatait és publikus metódusait bemutató osztálydiagram(ok). Tipikus hibalehetőségek: csillag-topológia, szigetek.}
%
%\begin{figure}[h]
%\begin{center}
%%\includegraphics[width=17cm]{chapters/chapter03/example.pdf}
%\caption{x}
%\label{fig:example1}
%\end{center}
%\end{figure}

\section{Szekvencia diagramok}
%\comment{Inicializálásra, use-case-ekre, belső működésre. Konzisztens kell legyen az előző alfejezettel. Minden metódus, ami ott szerepel, fel kell tűnjön valamelyik szekvenciában. Minden metódusnak, ami szekvenciában szerepel, szereplnie kell a valamelyik osztálydiagramon.}
\subsection{Akadály elhelyezése}
\begin{figure}[H]
\begin{center}
\includegraphics[width=17cm]{chapters/chapter03/images/Akadaly_elhelyezese.jpg}
\caption{Akadály elhelyezése szekvenciadiagram}
\label{fig:Akadály_elhelyezése}
\end{center}
\end{figure}

\subsection{Ellenfél mozgása}
\begin{figure}[H]
\begin{center}
\includegraphics[width=17cm]{chapters/chapter03/images/Ellenfel_mozgasa.jpg}
\caption{Ellenfél mozgása szekvenciadiagram}
\label{fig:Ellenfél_mozgása}
\end{center}
\end{figure}

\subsection{Torony eladása}
\begin{figure}[H]
\begin{center}
\includegraphics[width=17cm]{chapters/chapter03/images/Torony_eladasa.jpg}
\caption{Torony eladása szekvenciadiagram}
\label{fig:Torony_eladása}
\end{center}
\end{figure}

\subsection{Torony elhelyezése}
\begin{figure}[H]
\begin{center}
\includegraphics[width=17cm]{chapters/chapter03/images/Torony_elhelyezese.jpg}
\caption{Torony elhelyezése szekvenciadiagram}
\label{fig:Torony_elhelyezése}
\end{center}
\end{figure}

\subsection{Torony fejlesztése}
\begin{figure}[H]
\begin{center}
\includegraphics[width=17cm]{chapters/chapter03/images/Torony_fejlesztese.jpg}
\caption{Torony fejlesztése szekvenciadiagram}
\label{fig:Torony_fejlesztése}
\end{center}
\end{figure}

\subsection{Torony tüzelése}
\begin{figure}[H]
\begin{center}
\includegraphics[width=17cm]{chapters/chapter03/images/Torony_tuzelese.jpg}
\caption{Torony tüzelése szekvenciadiagram}
\label{fig:Torony_tüzelése}
\end{center}
\end{figure}


\section{State-chartok}
%\comment{Csak azokhoz az osztályokhoz, ahol van értelme. Egyetlen állapotból álló state-chartok ne szerepeljenek. A játék működését bemutató state-chart-ot készíteni tilos.}
\subsection{Field állapota}
\begin{figure}[H]
\begin{center}
\includegraphics[width=17cm]{chapters/chapter03/images/StatechartDiagram_Field.jpg}
\caption{Field állapotdiagram}
\label{fig:Field_állapotdiagram}
\end{center}
\end{figure}

\subsection{Path állapota}
\begin{figure}[H]
\begin{center}
\includegraphics[width=17cm]{chapters/chapter03/images/StatechartDiagram_Path.jpg}
\caption{Path állapotdiagram}
\label{fig:Path_állapotdiagram}
\end{center}
\end{figure}
