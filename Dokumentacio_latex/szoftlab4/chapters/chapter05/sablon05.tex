% Szglab4
% ===========================================================================
%
\chapter{Szkeleton tervezése}

\thispagestyle{fancy}

\section{A szkeleton modell valóságos use-case-ei}
%\comment{A szkeletonnak, mint önálló programnak a működésével kapcsolatos use-case-ek.}

\subsection{Use-case diagram}

\subsubsection{1. use-case diagram}
\begin{figure}[H]
\begin{center}
\includegraphics[width=17cm]{chapters/chapter05/images/user_uc_1.jpg}
\caption{Játékos use-case diagram 1.}
\label{fig:SzkeletonUseCase1}
\end{center}
\end{figure}


\subsubsection{2. use-case diagram}
\begin{figure}[H]
\begin{center}
\includegraphics[width=17cm]{chapters/chapter05/images/user_uc_2.jpg}
\caption{Játékos use-case diagram 2.}
\label{fig:SzkeletonUseCase1}
\end{center}
\end{figure}

\subsection{Use-case leírások}
%\comment{Minden use-case-hez külön}

\usecase
{1. Akadály elhelyezése ellenség által foglalt útra}
{A játékos egy akadályt akar helyezni egy útra, amin már van egy ellenség}
{User}
{Lekérdezés, hogy az úton van-e ellenség vagy akadály, van ellenség, nem történik semmi.}

\usecase
{2. Akadály elhelyezése másik akadály által foglalt útra}
{A játékos egy akadályt akar helyezni egy útra, amin már van egy akadály}
{User}
{Lekérdezés, hogy az úton van-e ellenség vagy akadály, van akadály, nem történik semmi.}

\usecase
{3. Akadály elhelyezése üres útra}
{A játékos egy akadályt akar helyezni egy útra, amin nincs sem ellenség sem akadály}
{User}
{Lekérdezés, hogy az úton van-e ellenség vagy akadály, nincs egyik sem, varázserő levonása, akadály létrehozása, akadály elhelyezése az úton.}

\usecase
{4. Akadály fejlesztése}
{Akadály fejlesztése minden gem-el}
{User}
{Minden, az akadályhoz tartozó gem típusból létrehozunk egyet, hozzáadjuk az akadályhoz és érvényre juttatjuk a fejlesztést}

\usecase
{5. Ellenség mozog}
{Egy ellenség egy celláról egy másikra jut}
{User}
{Az ellenség elérhet a végzet hegyéhez, egyébként a következő cellára lép}

\usecase
{6. Fejlesztetlen torony eladása}
{A játékos elad egy tornyot}
{User}
{A tornyot töröljük a mezőjéről és a Game osztály adott példányából, manát növeljük}

\usecase
{7. Fejlesztett torony eladása}
{A Játékos elad egy fejlesztett tornyot}
{User}
{A tornyot töröljük a mezőjéről és a Game osztály adott példányából. A toronyban az összes típusú gemből van egy. Az ezek által adott adatokból és a torony árából képzett értékkel módosítjuk a manát.}

\usecase
{8. Inicializálás}
{A játék egy kezdeti állapotának felvétele}
{User}
{A cellák létrehozása, pálya betöltése, kezdeti értékek beállítása }

\usecase
{9. Torony elhelyezése foglalt mezőre}
{A játékos egy tornyot akar helyezni egy mezőre, amin már van másik torony}
{User}
{Foglaltság ellenőrzése, mező foglalt, nem történik semmi.}

\usecase
{10. Torony elhelyezése üres mezőre}
{A játékos egy tornyot akar helyezni egy mezőre, amin még nincs másik torony}
{User}
{Foglaltság ellenőrzése, varázserő mennyiségének ellenőrzése, mindkettő rendben, varázserő levonása, torony létrehozása, torony regisztrálása a game-ben, torony elhelyezése a mezőn.}

\usecase
{11. Torony fejlesztése}
{Torony fejlesztése minden gem típusból eggyel}
{User}
{A toronyhoz hozzáadjuk a gem példányokat, mindegyik fejlesztését érvényre juttatjuk}

\usecase
{12. Torony hatókörében nincs ellenség tüzeléskor}
{Torony hatókörében nincs ellenség tüzeléskor}
{User}
{A torony elkéri az ellenséget és nem kap ellenséget. Nem csinál semmit}

\usecase
{13. Torony lelő egy ellenséget, fejlesztetlenül}
{Torony lelő egy ellenséget, fejlesztetlenül}
{User}
{A torony elkéri a célpontot, sebez rajta az alapértékekkel, mivel nincs fejlesztve, az ellenség meghal, ezért töröljük az útról és a Game példányból. Ugyanígy működne fejlesztett esetben.}

\usecase
{14. Torony tüzel egy ellenségre, fejlesztetlenül}
{Torony tüzel egy ellenségre, fejlesztetlenül}
{User}
{A torony elkéri a célpontot, sebez rajta az alapértékekkel, mivel nincs fejlesztve, az ellenség nem hal meg, csak sebződik. Ugyanígy működne fejlesztett esetben.}




\section{A szkeleton kezelői felületének terve, dialógusok}
%\comment{A szkeleton által elfogadott bemenetek , valamint a szöveges konzolon megjelenő kimenetek. A kiemenet formátuma olyan kell legyen, ami alapján a működés összevethető a korábbi szekvencia-diagramokkal.}
\subsection{Kezelőfelület}
A szkeleton arra alkalmas, hogy egyszerűen tudjuk tesztelni, hogy a szekvencia diagramoknak megfelelően hívódnak-e meg a metódusok.

Tényleges számítások, algoritmusok nem szerepelnek a programban, ez még csak az osztályok közötti kommunikációt teszteli.

A program előre megírt teszteseteket tud majd futtatni, amik terveink szerint lefedik a program működésének lényegi részét.
Konzolon, karakteres bevitellel lehet az egyes teszteseteket kiválasztani. 

Minden objektum függvénye hívásakor kiír egy nyilat ($\rightarrow$),kiírja a nevét, és a függvény nevét.

Ha újabb függvényt hív, egy tabulátorral beljebb írja ki.

Visszatéréskor egy visszafelé mutató nyilat ($\leftarrow$) és, hogy melyik függvényből tér vissza.

\subsection{Tesztesetek}
%
% Teszteset
%
\newcommand{\usecaseteszteset}[4]
{
	\begin{longtable}{| l | p{10cm} |}
	\hline
	\textbf{Teszt-eset neve}   & {\textbf{#1}} \tabularnewline
	\hline\hline
	Rövid leírás    & {#2} \tabularnewline
	\hline
	Használt interfészek    & {#3} \tabularnewline
	\hline
	Tesztelt use-case	& {#4} \tabularnewline
	\hline
	\end{longtable}
}

\usecaseteszteset
{1. ObstacleBuyOnEnemy}
{Akadály elhelyezése olyan mezőre amin ellenség van}
{}
{Akadály elhelyezése ellenség által foglalt útra}

\usecaseteszteset
{2. ObstacleBuyOnObstacle}
{Akadály elhelyezése másik akadály által foglalt útra}
{}
{Akadály elhelyezése másik akadály által foglalt útra}

\usecaseteszteset
{3. ObstacleBuy}
{Akadály elhelyezése üres útra}
{}
{Akadály elhelyezése üres útra}

\usecaseteszteset
{4. ObstacleBuyGems}
{Akadály feljesztése}
{}
{Akadály feljesztése}

\usecaseteszteset
{5. EnemyMove}
{Ellenség léptetése}
{IPathPlaceable}
{Ellenség mozog 3 cellát, akadály nélkül, esetleg bejut a végzet hegyéhez}

\usecaseteszteset
{6. TowerSellNonUpgraded}
{Torony eladása}
{ITower, IGame}
{Fejlesztetlen torony eladása}

\usecaseteszteset
{7. TowerSellUpgraded}
{Torony eladása}
{ITower, IGame}
{Fejlesztett torony eladása}

\usecaseteszteset
{8. Initialize}
{A játék kezdeti betöltésének lefuttatása}
{}
{Inicializálás}
 
\usecaseteszteset
{9. TowerBuyOnField}
{Egy torony elhelyezése a pályán}
{IGame, IFieldPlaceable}
{Torony elhelyezése foglalt mezőre}

\usecaseteszteset
{10. TowerBuy}
{Egy torony elhelyezése a pályán}
{IGame, IFieldPlaceable}
{Torony elhelyezése üres mezőre}

\usecaseteszteset
{11. TowerBuyGems}
{Torony feljesztése minden gemből eggyel}
{ITower}
{Torony feljesztése minden gemből eggyel}

\usecaseteszteset
{12. TowerShootNoEnemy}
{Torony hatókörében nincs ellenség tüzeléskor}
{ITower}
{Torony hatókörében nincs ellenség tüzeléskor}

\usecaseteszteset
{13. TowerKillEnemy}
{Torony megöl egy ellenséget}
{ITower}
{Torony lelő egy ellenséget, fejlesztetlenül}

\usecaseteszteset
{14. TowerShootEnemy}
{Fejlesztetlen torony, hatósugarában egy ellenség van}
{ITower}
{Torony tüzel egy ellenségre, fejlesztetlenül}





\section{Szekvencia diagramok a belső működésre}
%\comment{A szkeletonban implementált szekvenciadiagramok. Tipikusan egy use-case egy diagram. Ezek megegyezhetnek a korábban specifikált diagramokkal, de az egyes életvonalakat (lifeline) egyértelműen a szkeletonban példányosított objektumokhoz kell tudni kötni. Azt kell megjeleníteni, hogy a szkeletonban létrehozott objektumok egymással hogyan fognak kommunikálni.}

\subsection{1. Akadály elhelyezése ellenség által foglalt útra}
\begin{figure}[H]
\begin{center}
\includegraphics[width=17cm]{chapters/chapter05/images/sd_Akadaly_elhelyezese_ellenseg_altal_foglalt_utra.jpg}
\caption{Akadály elhelyezése ellenség által foglalt útra szekvenciadiagram}
\label{fig:sd_Akadaly_elhelyezese_ellenseg_altal_foglalt_utra}
\end{center}
\end{figure}

\subsection{2. Akadály elhelyezése másik akadály által foglalt útra}
\begin{figure}[H]
\begin{center}
\includegraphics[width=17cm]{chapters/chapter05/images/sd_Akadaly_elhelyezese_masik_akadaly_altal_foglalt_utra.jpg}
\caption{Akadály elhelyezése másik akadály által foglalt útra szekvenciadiagram}
\label{fig:sd_Akadaly_elhelyezese_masik_akadaly_altal_foglalt_utra}
\end{center}
\end{figure}

\subsection{3. Akadály elhelyezése üres útra}
\begin{figure}[H]
\begin{center}
\includegraphics[width=17cm]{chapters/chapter05/images/sd_Akadaly_elhelyezese_ures_utra.jpg}
\caption{Akadály elhelyezése üres útra szekvenciadiagram}
\label{fig:sd_Akadaly_elhelyezese_ures_utra}
\end{center}
\end{figure}

\subsection{4. Akadály fejlesztése}
\begin{figure}[H]
\begin{center}
\includegraphics[width=17cm]{chapters/chapter05/images/sd_Akadaly_fejlesztese.jpg}
\caption{Akadály fejlesztése szekvenciadiagram}
\label{fig:sd_Akadaly_fejlesztese}
\end{center}
\end{figure}

\subsection{5. Ellenfél mozgása}
\begin{figure}[H]
\begin{center}
\includegraphics[width=17cm]{chapters/chapter05/images/sd_Ellenfel_mozgasa.jpg}
\caption{Ellenfél mozgása szekvenciadiagram}
\label{fig:sd_Ellenfel_mozgasa}
\end{center}
\end{figure}

\subsection{6. Fejlesztetlen torony eladása}
\begin{figure}[H]
\begin{center}
\includegraphics[width=17cm]{chapters/chapter05/images/sd_Fejlesztetlen_torony_eladasa.jpg}
\caption{Fejlesztetlen torony eladása szekvenciadiagram}
\label{fig:sd_Fejlesztetlen_torony_eladasa}
\end{center}
\end{figure}

\subsection{7. Fejlesztett torony eladása}
\begin{figure}[H]
\begin{center}
\includegraphics[width=17cm]{chapters/chapter05/images/sd_Fejlesztett_torony_eladasa.jpg}
\caption{Fejlesztett torony eladása szekvenciadiagram}
\label{fig:sd_Fejlesztett_torony_eladasa}
\end{center}
\end{figure}

\subsection{8. Inicializálás}
\begin{figure}[H]
\begin{center}
\includegraphics[width=17cm]{chapters/chapter05/images/sd_Inicializalas.jpg}
\caption{Inicializálás szekvenciadiagram}
\label{fig:sd_Inicializalas}
\end{center}
\end{figure}

\subsection{9. Torony elhelyezése foglalt mezőre}
\begin{figure}[H]
\begin{center}
\includegraphics[width=17cm]{chapters/chapter05/images/sd_Torony_elhelyezese_foglalt_mezore.jpg}
\caption{Torony elhelyezése foglalt mezőre szekvenciadiagram}
\label{fig:sd_Torony_elhelyezese_foglalt_mezore}
\end{center}
\end{figure}

\subsection{10. Torony elhelyezése üres mezőre}
\begin{figure}[H]
\begin{center}
\includegraphics[width=17cm]{chapters/chapter05/images/sd_Torony_elhelyezese_ures_mezore.jpg}
\caption{Torony elhelyezése üres mezőre szekvenciadiagram}
\label{fig:sd_Torony_elhelyezese_ures_mezore}
\end{center}
\end{figure}

\subsection{11. Torony fejlesztése minden gemből eggyel}
\begin{figure}[H]
\begin{center}
\includegraphics[width=17cm]{chapters/chapter05/images/sd_Torony_fejlesztese_minden_gembol_eggyel.jpg}
\caption{Torony fejlesztése minden gemből eggyel szekvenciadiagram}
\label{fig:sd_Torony_fejlesztese_minden_gembol_eggyel}
\end{center}
\end{figure}

\subsection{12. Torony hatókörében nincs ellenség tüzeléskor}
\begin{figure}[H]
\begin{center}
\includegraphics[width=17cm]{chapters/chapter05/images/sd_Torony_hatokoreben_nincs_ellenseg_tuzeleskor.jpg}
\caption{Torony hatókörében nincs ellenség tüzeléskor szekvenciadiagram}
\label{fig:sd_Torony_hatokoreben_nincs_ellenseg_tuzeleskor}
\end{center}
\end{figure}

\subsection{13. Torony lelő egy ellenséget fejlesztetlenül}
\begin{figure}[H]
\begin{center}
\includegraphics[width=17cm]{chapters/chapter05/images/sd_Torony_lelo_egy_ellenseget_fejlesztetlenul.jpg}
\caption{Torony lelő egy ellenséget fejlesztetlenül szekvenciadiagram}
\label{fig:sd_Torony_lelo_egy_ellenseget_fejlesztetlenul}
\end{center}
\end{figure}

\subsection{14. Torony tüzel egy ellenségre fejlesztetlenül}
\begin{figure}[H]
\begin{center}
\includegraphics[width=17cm]{chapters/chapter05/images/sd_Torony_tuzel_egy_ellensegre_fejlesztetlenul.jpg}
\caption{Torony tüzel egy ellenségre fejlesztetlenül szekvenciadiagram}
\label{fig:sd_Torony_tuzel_egy_ellensegre_fejlesztetlenul}
\end{center}
\end{figure}



\section{Kommunikációs diagramok}
%\comment{A szkeletonban, az egyes szkeleton-use-case-ek futása során létrehozott objektumok és kapcsolataik bemutatására szolgáló diagramok. Ezek alapján valósítják meg a szkeleton fejlesztői az inicializáló kódrészleteket.}

\subsection{1. Akadály elhelyezése ellenség által foglalt útra}
\begin{figure}[H]
\begin{center}
\includegraphics[width=17cm]{chapters/chapter05/images/cd_Akadaly_elhelyezese_ellenseg_altal_foglalt_utra.jpg}
\caption{Akadály elhelyezése ellenség által foglalt útra kommunikáviós diagram}
\label{fig:cd_Akadaly_elhelyezese_ellenseg_altal_foglalt_utra}
\end{center}
\end{figure}

\subsection{2. Akadály elhelyezése másik akadály által foglalt útra}
\begin{figure}[H]
\begin{center}
\includegraphics[width=17cm]{chapters/chapter05/images/cd_Akadaly_elhelyezese_masik_akadaly_altal_foglalt_utra.jpg}
\caption{Akadály elhelyezése másik akadály által foglalt útra kommunikáviós diagram}
\label{fig:cd_Akadaly_elhelyezese_masik_akadaly_altal_foglalt_utra}
\end{center}
\end{figure}

\subsection{3. Akadály elhelyezése üres útra}
\begin{figure}[H]
\begin{center}
\includegraphics[width=17cm]{chapters/chapter05/images/cd_Akadaly_elhelyezese_ures_utra.jpg}
\caption{Akadály elhelyezése üres útra kommunikáviós diagram}
\label{fig:cd_Akadaly_elhelyezese_ures_utra}
\end{center}
\end{figure}

\subsection{4. Akadály fejlesztése}
\begin{figure}[H]
\begin{center}
\includegraphics[width=17cm]{chapters/chapter05/images/cd_Akadaly_fejlesztese.jpg}
\caption{Akadály fejlesztése kommunikáviós diagram}
\label{fig:cd_Akadaly_fejlesztese}
\end{center}
\end{figure}

\subsection{5. Ellenfél mozgása}
\begin{figure}[H]
\begin{center}
\includegraphics[width=17cm]{chapters/chapter05/images/cd_Ellenfel_mozgasa.jpg}
\caption{Ellenfél mozgása kommunikáviós diagram}
\label{fig:cd_Ellenfel_mozgasa}
\end{center}
\end{figure}

\subsection{6. Fejlesztetlen torony eladása}
\begin{figure}[H]
\begin{center}
\includegraphics[width=17cm]{chapters/chapter05/images/cd_Fejlesztetlen_torony_eladasa.jpg}
\caption{Fejlesztetlen torony eladása kommunikáviós diagram}
\label{fig:cd_Fejlesztetlen_torony_eladasa}
\end{center}
\end{figure}

\subsection{7. Fejlesztett torony eladása}
\begin{figure}[H]
\begin{center}
\includegraphics[width=17cm]{chapters/chapter05/images/cd_Fejlesztett_torony_eladasa.jpg}
\caption{Fejlesztett torony eladása kommunikáviós diagram}
\label{fig:cd_Fejlesztett_torony_eladasa}
\end{center}
\end{figure}

\subsection{8. Inicializálás}
\begin{figure}[H]
\begin{center}
\includegraphics[width=17cm]{chapters/chapter05/images/cd_Inicializalas.jpg}
\caption{Inicializálás kommunikáviós diagram}
\label{fig:cd_Inicializalas}
\end{center}
\end{figure}

\subsection{9. Torony elhelyezése foglalt mezőre}
\begin{figure}[H]
\begin{center}
\includegraphics[width=17cm]{chapters/chapter05/images/cd_Torony_elhelyezese_foglalt_mezore.jpg}
\caption{Torony elhelyezése foglalt mezőre kommunikáviós diagram}
\label{fig:cd_Torony_elhelyezese_foglalt_mezore}
\end{center}
\end{figure}

\subsection{10. Torony elhelyezése üres mezőre}
\begin{figure}[H]
\begin{center}
\includegraphics[width=17cm]{chapters/chapter05/images/cd_Torony_elhelyezese_ures_mezore.jpg}
\caption{Torony elhelyezése üres mezőre kommunikáviós diagram}
\label{fig:cd_Torony_elhelyezese_ures_mezore}
\end{center}
\end{figure}

\subsection{11. Torony fejlesztése minden gemből eggyel}
\begin{figure}[H]
\begin{center}
\includegraphics[width=17cm]{chapters/chapter05/images/cd_Torony_fejlesztese_minden_gembol_eggyel.jpg}
\caption{Torony fejlesztése minden gemből eggyel kommunikáviós diagram}
\label{fig:cd_Torony_fejlesztese_minden_gembol_eggyel}
\end{center}
\end{figure}

\subsection{12. Torony hatókörében nincs ellenség tüzeléskor}
\begin{figure}[H]
\begin{center}
\includegraphics[width=17cm]{chapters/chapter05/images/cd_Torony_hatokoreben_nincs_ellenseg_tuzeleskor.jpg}
\caption{Torony hatókörében nincs ellenség tüzeléskor kommunikáviós diagram}
\label{fig:cd_Torony_hatokoreben_nincs_ellenseg_tuzeleskor}
\end{center}
\end{figure}

\subsection{13. Torony lelő egy ellenséget fejlesztetlenül}
\begin{figure}[H]
\begin{center}
\includegraphics[width=17cm]{chapters/chapter05/images/cd_Torony_lelo_egy_ellenseget_fejlesztetlenul.jpg}
\caption{Torony lelő egy ellenséget fejlesztetlenül kommunikáviós diagram}
\label{fig:cd_Torony_lelo_egy_ellenseget_fejlesztetlenul}
\end{center}
\end{figure}

\subsection{14. Torony tüzel egy ellenségre fejlesztetlenül}
\begin{figure}[H]
\begin{center}
\includegraphics[width=17cm]{chapters/chapter05/images/cd_Torony_tuzel_egy_ellensegre_fejlesztetlenul.jpg}
\caption{Torony tüzel egy ellenségre fejlesztetlenül kommunikáviós diagram}
\label{fig:cd_Torony_tuzel_egy_ellensegre_fejlesztetlenul}
\end{center}
\end{figure}








