% Szglab4
% ===========================================================================
%
\chapter{Analízis modell kidolgozása 2}

\thispagestyle{fancy}

\section{Objektum katalógus}

%\comment{Minden, a feladatban szereplő objektum rövid, egy-két bekezdés hosszú ismertetése. Meg kell jelenjen minden objektumhoz, hogy mi a felelőssége. Informális leírás, ezért nem kell foglalkozni az örökléssel, az interfészekkel, az absztrakt osztályokkal, a segédosztályokkal.}

%\subsection{Objektum1}
%\comment{Felelősség informális leírása}
%
%\subsection{Objektum2}
%\comment{Felelősség informális leírása}



\subsection{Akadály (Obstacle)}
Az akadály (Obstacle) objektum felelőssége egyrészt az, hogy amikor áthalad rajta egy ellenség (Enemy) lelassítsa. Másfelől felelőssége az is, hogy egy-egy ellenség  áthaladtával amortizálódjon, valamint ha már teljesen elhasználódott, értesítse azt az út elemet (Path), amelyiken áll.
\subsection{Ellenség (Enemy)}
Egy ellenséget (tünde, hobbit, törp vagy ember) megvalósító objektum. Az ő felelőssége, hogy egy adott lövedék (Bullet) hatására sebződjön, vagy ha már sokat sebződött, akkor haljon meg, valamint az, hogy celláról cellára mozgassa magát, és ha eléri a végzet hegyét értesítse a játék (Game) osztályt, hogy intézkedjen.

\subsection{Játék (Game)}
A játék (Game) objektum felelőssége többek közt a játék ütemezése, az idő múlásának kontrollálása. Ezenkívül az inicializálás, vagyis a játék kezdeti állapotának felvétele, továbbá a modell állapotának folyamatos változása miatti frissítés, valamint ennek a grafikus felületen való megjelenítése. A game osztály hozza létre az ellenségeket és indítja el őket az úton.

\subsection{Kristály (Gem)}
	Ha a játékos vesz a toronyra/akadályra valamilyen kristályt, akkor jön létre, megkapja a torony, és beépíti magába. Felelőssége, hogy általa érvényre jussanak a fejlesztések.
\subsection{Lövedék (Bullet)}
	A tornyok egy lövedéket tárolnak, amit minden lövésnél átadnak a lövő függvénynek. Ennek a lövedéknek a feladata, hogy az ellenségnek megmondja mennyit sebez rajta.
\subsection{Mező (Field)}
A Field osztály a Cell osztály leszérmazottja. A nem út típusú cellákat (mező) reprezentálja. Egy mezőre egy torony helyezhető. 

\subsection{Pálya (Map)}
A Map osztály a játéktér elemeit, mint cellák tárolja, egy két dimenziós tömbben. Megadja minden egyes cellához, a szomszédjai referenciáját. 




\subsection{Torony (Tower)}
	Az egyetlen tervezett toronytípus, le lehet rakni a pályán az úton kívül bárhova. A hatósugarába belépett ellenségekre lőnie kell, lehet fejleszteni lövési sebességét, erejét, újratöltési idejét és egy ellenségtípusra még hatásosabbá tenni a lövedékeit. A játékos varázserőért tud lerakni, illetve eladni tornyokat. Ez a legfontosabb eszköz amivel a játékos meg tudja akadályozni az ellenségek célbajutását.
\subsection{Út (Path)}
A Path a Cell osztály leszármazottja. Az út típusú cellákat reprezentálja. Tartalmazza a rajta lévő ellenségeket és esetleg akadályt. Minden út tudja azt is, hogy hova lehet lépni róla egy lépésben.

\subsection{Obstaclehez tartozó kristályok: Intensity, Repair}
Akadályra helyezhető kristályok, ami növeli a lassítás értékét vagy megjavítja az akadályt.

\subsection{Towerhez tartozó krsitályok: Range, Speed, Damage, EnemyType}
A torony lövésének hatósugarát, gyorsítását, sebzését, ellenféltípusra sebzés, növelő kristályok osztályai.

\subsection{ITGem, IOGem}
Ezen keresztül lehet kezelni a kristályokat, és a tornyok/akadályok fejlesztését. Eladáshoz ITGem-nek felelőssége, hogy az árát a kristálynak visszaadja.




\section{Statikus struktúra diagramok}
%\comment{Az előző alfejezet osztályainak kapcsolatait és publikus metódusait bemutató osztálydiagram(ok). Tipikus hibalehetőségek: csillag-topológia, szigetek.}

\begin{figure}[H]
\begin{center}
\includegraphics[width=17cm]{chapters/chapter04/images/Main.jpg}
\caption{Osztálydiagram}
\label{fig:Osztálydiagram}
\end{center}
\end{figure}


\section{Osztályok leírása}
%\comment{Az előző alfejezetben tárgyalt objektumok felelősségének formalizálása attribútumokká, metódusokká. Csak publikus metódusok szerepelhetnek. Ebben az alfejezetben megjelennek az interfészek, az öröklés, az absztrakt osztályok. Segédosztályokra még mindig nincs szükség. Az osztályok ABC sorrendben kövessék egymást. Interfészek esetén az Interfészek, Attribútumok pontok kimaradnak.}

\subsection{Bullet}
\begin{itemize}
\item Felelősség\\
Ellenség kapja meg, és ebből tudja meg mennyire sebződik.
%\comment{Mi az osztály felelőssége. Kb 1 bekezdés.}
\item Ősosztályok\\
Object
%\comment{Mely osztályokból származik (öröklési hierarchia)\newline
%Legősebb osztály $\rightarrow$ Ősosztály2 $\rightarrow$ Ősosztály3...}
\item Interfészek\\
Nincs
%\comment{Mely interfészeket valósítja meg.}
\item Attribútumok\\
%\comment{Milyen attribútumai vannak}
	\begin{description}
		\item[int damage] alapsebzés
		\item[Enemy enemyType] a torony itt tárolja, hogy melyik ellenség típusra erősebb a sebzése
	\end{description}
\item Metódusok\\
%\comment{Milyen publikus metódusokkal rendelkezik. Metódusonként egy-három mondat arról, hogy a metódus mit csinál.}
	\begin{description}
		\item[Bullet(int damage, Enemy enemyType)] Konstruktor
\item[int getHobbitDamage()] ha hobbitot sebez, ezzel a függvénnyel kérdezi le a sebzés értékét. 
\item[int getHumanDamage()] ha embert sebez, ezzel a függvénnyel kérdezi le a sebzés értékét.
\item[int getDwarfDamage()] ha törpöt sebez, ezzel a függvénnyel kérdezi le a sebzés értékét.
\item[int getElfDamage()] ha tündét sebez, ezzel a függvénnyel kérdezi le a sebzés értékét
\item[void setEnemy(Enemy e)] beállítja az ellenséget akire specializált
\item[void setDamage(int damage)] beállítja a lövedék sebzését

	\end{description}
\end{itemize}


\subsection{Enemy}
\begin{itemize}
\item Felelősség\\
Tudja, hogy mennyi élete van még, milyen sebességgel haladt eredetileg, és milyen sebességgel halad most. Ez egy absztrakt ősosztály, ami összefogja a 4 ellenségtípust (Hobbit, Elf, Dwarf, Human).
\item Ősosztályok\\
Object
\item Interfészek\\
IPathPlaceable
\item Attribútumok\\
	\begin{description}
		\item[int speed] A két lépés között eltelt idő.
\item[int modSpeed] Az ellenség belső idő mérője. A setModSpeed változtathatja – jellemzően negatív irányba, akadályokon.
\item[int health] Életerejét tárolja ebben. Hurt függvényben csökkenti.
\item[Path myPath] az a mező, ahol tartózkodik
\item[IGame igame] ezen keresztül tudja módosítani a manát, amikor meghal, illetve ha elér a végzet hegyére módosítani a számlálót (Game.succeededE), hogy nőjön egyel

	\end{description}
\item Metódusok\\
	\begin{description}
		\item[hurt(Bullet b)] sebződik (abstract method)
\item[move()] mozog, a következő path-ra lép, cellát vált
\item[Enemy(int sp, int msp, int h, ig: IGame)] konstruktor
\item[void setModSpeed(int msp)] modSpeed változót változtatja. Lassítani lehet vele.

	\end{description}
\end{itemize}

\subsection{Enemy subclasses: Elf, Hobbit, Dwarf, Human}
\begin{itemize}
\item Felelősség\\
Sebződés: egy Bullet alapján a saját életét csökkenteni, és ha kell, meghalni. Tehát felüldefiniálja az Enemy ősosztály hurt metódusát.
\item Ősosztályok\\
Object $\rightarrow$ Enemy
\item Interfészek\\
IPathPlaceable

\item Metódusok\\
	\begin{description}
		\item[hurt(Bullet)] sebződik, a kapott Bullet alapján bizonyos mértékű összeget levon az életpontjukból.
		
	\end{description}
\end{itemize}



\subsection{Game}
\begin{itemize}
\item Felelősség\\
Lásd objektum katalógus.
\item Ősosztályok\\
Object
\item Interfészek\\
Nincs
\item Attribútumok\\
	\begin{description}
		\item[Map map] játék térképe
\item[List<Enemy> enemiesOut] pályára még be nem lépett ellenségek
\item[List<Enemy> enemiesIn] pályára már belépett ellenségek
\item[List<Tower> towers] a tornyok, amik a pályán vannak
\item[int mana] maradék varázserő
\item[Path firstP] az út kezdő cellája
\item[int noEnemies] kezdeti hullámérték, amely folyamatosan nő, azt mutatja meg, hogy következő körben hány ellenséget kell létrehozni és beküldeni a pályára
\item[int succeeded] végzet hegyét elért enemy-k száma



		
	\end{description}
\item Metódusok\\
	\begin{description}
		\item[void update()] frissíti a modellt, grafikát
\item[void initialize(String name)] kiinduló állapot felvétele
\item[Game()] konstruktor
\item[void makeEnemies()] létrehoz néhány ellenséget, ezeket beteszi az enemiesIn-be

		
		
	\end{description}
\end{itemize}
\subsection{IGame}
\begin{itemize}
\item Felelősség\\
Az IGame interfész szolgáltatást nyújt az akadályoknak, tornyoknak, ellenségeknek, hogy rajta keresztül manát írjanak jóvá/csökkentsenek, illetve ellenségek esetén a végzet hegyét elért ellenségek számát módosítsák. Speciális interfész a Game osztályhoz. 

\item Metódusok\\
	\begin{description}
		\item[void changeMana()] manát megváltoztató metódus
\item[int incSucceeded()] succeededE értékét megváltoztató metódus
\item[void addTower(Tower t)] hozzáad egy tornyot a listához
\item[void deleteEnemy(Enemy e)] ellenség törlése enemiesIn-ből
\item[void deleteTower(Tower t)] torony törlése towers-ből

		
	\end{description}
\end{itemize}

\subsection{Gem}
\begin{itemize}
\item Felelősség\\
A kristály osztály felel a torony fejlesztéséért.
\item Ősosztályok\\
Object
\item Interfészek\\
Nincs
\item Attribútumok\\
	\begin{description}
		\item[static int price] az ár, amennyi varázserőbe kerül.
	\end{description}
\item Metódusok\\
	\begin{description}
		
		\item[Gem(int price)] konstruktor
		
		
	\end{description}
\end{itemize}

\subsection{IObstacle}
\begin{itemize}
\item Felelősség\\
Olyan metódusok használatát teszi lehetővé, amelyek az Obstacle típusú elemek viselkedését modellezik
\item Ősosztályok\\
Nincs

\item Metódusok\\
	\begin{description}
		\item[void slow(int intesity, Path p)] szól p-nek, hogy lassítsa le az ellenséget intensity-vel
\item[void amortization()] amortizál
\item[void increaseIntesity(int intens)] megnöveli az intesity-t intens-el
\item[void addIOGem(IOGem: iog)] iog kristályt hozzáadja az akadályhoz
\item[repair()] megjavítja az akadályt


		
		
	\end{description}
\end{itemize}
\subsection{Obstacle}
\begin{itemize}
\item Felelősség\\
Lásd objektum katalógus. 
\item Ősosztályok\\
Nincs
\item Interfészek\\
IObsacle, IPathPlaceable
\item Attribútumok\\
	\begin{description}
	\item[int slowIntens] lassítás mértéke
\item[Path myPath] a mező, amin rajta van
\item[int amort] az elhasználódottság mértéke
\item[static final int price] az ára
\item[ArrayList<IOGem> gems] a megvett kristályok listája


	\end{description}
\item Metódusok\\
	\begin{description}
		
		\item[Obstacle(int intens, Path p, int amort, int price, bool up)] konstruktor
		
		
	\end{description}
\end{itemize}
\subsection{ITower}
\begin{itemize}
\item Felelősség\\
A torony funkciói vannak benne.
\item Metódusok\\
	\begin{description}
		\item[void setPaths()] a saját cellájából kiindulva a hatósugarával lefedett területen felkeresi, és beregisztrálja a paths listába a path cellákat.
\item[void shoot()] A torony akkor lő, ha letelt az újratöltési idő, ekkor megnézi, hogy lőtávon belül van-e ellenség, és ha van meghívja a sebzés függvényét, átadva paraméterként a lövedékét. 
\item[void addITGem(ITGem gem)] paraméterként megkapja a kiválasztott kristályt, a gameStat-ot frissíti, és a bullet-et is.
\item[Enemy chooseEnemy()] A torony tárolja a hatókörbe eső path cellákat. Minden tick-ben végig megy rajtuk, és kiválaszt egyet, amelyiken van ellenség, és oda fog lőni. Az ellenséggel tér vissza.

		
		
	\end{description}
\end{itemize}
\subsection{Tower}
\begin{itemize}
\item Felelősség\\
Lásd objektumkatalógus
\item Ősosztályok\\
Nincs
\item Interfészek\\
ITower, IFieldPlaceable
\item Attribútumok\\
	\begin{description}
		\item[static final int price] az ára varázserőben.
\item[int range] lőtáv, hatókör.
\item[int speedCtr] A torony belső idő mérője. Ezt vizsgálja minden lövés előtt, hogy eltelt e elég idő.
\item[int speed] két lövés között eltelt minimális idő.
\item[Bullet bullet] A torony tárol egy lövedéket, mindig ezt lövi ki.
\item[ArrayList<ITGem> gems] A megvásárolt kristályokat tárolja.
\item[Field myField] a mezőt tárolja amin áll.
\item[ArrayList<Path> paths] Hatósugárba eső út cellák.
\item[Field myField] mező, amin áll.
\item[IGame igame] Egy interfész a játék logikára, amivel a bejutott ellenségek számát, és a varázserőt is lehet állítani.

	\end{description}
\item Metódusok\\
	\begin{description}
		\item[Tower(int rang, int pr)] konstruktor
\item[void upgradeSpeed(int sp)] fejleszti a lövési sebességét.
\item[void upgradeRange(int rng)] fejleszti a lőtávot.
\item[void upgradeEnemy(Enemy e)] egy ellenségtípusra növeli a sebzést.
\item[void upgradeDamage(int dmg)] növeli s sebzést.

		
		
	\end{description}
\end{itemize}
\subsection{Map}
\begin{itemize}
\item Felelősség\\
Ld. objektum katalógus
\item Ősosztályok\\
Nincs
\item Interfészek\\
Nincs
\item Attribútumok\\
	\begin{description}
		\item[String name] a pálya neve, egyben az azonosítója
		\item[int level] a pálya szintje
		\item[Array<Array<Cell>> grid] A cellákat tartalmazó 2 dimenziós tömb

		
	\end{description}
\item Metódusok\\
	\begin{description}
		
		\item[Map(string name)] az osztály konstruktora, a paraméterként megadott névvel rendelkező fájlból betölti a pálya térképét
\item[void load(string name)] megnyitja a paraméterként kapott nevű fájlt, és abból betölti a pálya celláinak tulajdonságait, felépíti a pályát.
\item[int getLevel()] visszaadja a pálya szintjét.
\item[String getName()] visszaadja a pálya nevét
\item[Path getFirstPath()] visszaadja a pálya belépési pontjának referenciáját

		
		
	\end{description}
\end{itemize}
\subsection{Cell}
\begin{itemize}
\item Felelősség\\
A Cell a pálya egy egységét reprezentáló osztály. Létrehozásakor megkapja a 4 szomszédja referenciáját. Maga a cella nem tudja, hogy hol van a térképen. A cella tárolja a rajta éppen tartózkodó ellenségek referenciáit. A Cell osztály absztrakt.
\item Ősosztályok\\
Nincs
\item Interfészek\\
Nincs
\item Attribútumok\\
	\begin{description}
		\item[Array<Cell> neighbours] 4 elemű tömb, tárolja 4 irányban a szomszédjai referenciáját.

		
	\end{description}
\item Metódusok\\
	\begin{description}
		
		\item[Cell(Array<Cell>)] konstruktor, paraméterként kapja a szomszédos mezők referenciáit.
		\item[bool isPath()] olyan értékkel tér vissza amilyen típusú a cella 
		
		
	\end{description}
\end{itemize}
\subsection{Field}
\begin{itemize}
\item Felelősség\\
Ld. objektum katalógus
\item Ősosztályok\\
Object $\rightarrow$ Cell
\item Interfészek\\
Nincs
\item Attribútumok\\
	\begin{description}
		\item[ITower itower] a mezőn álló torony interfészű elem tárolása

		
	\end{description}
\item Metódusok\\
	\begin{description}
		
		\item[bool isPath()] hamis értékkel tér vissza
		\item[void addIFieldPlaceable()] egy új tornyot ad hozzá a mezőhöz
		\item[void deleteIFieldPlaceable(IFieldPlaceable ifield)] eltávolítja a tornyot a mezőről
		\item[void registerITower(itower ITower)] beteszi ifieldbe a kapott tornyot
		\item[Field()] konstruktor
		
		
	\end{description}
\end{itemize}



\subsection{Path}
\begin{itemize}
\item Felelősség\\
Ld objektum katalógus
\item Ősosztályok\\
Object $\rightarrow$ Cell
\item Interfészek\\
Nincs
\item Attribútumok\\
	\begin{description}
		\item[IObstacle iobstacle] az esetleg az úton levő akadályt tárolja
		\item[ArrayList<Enemy> enemies] az éppen áthaladó ellenségek listája
		\item[ArrayList<Path> paths] következő path-ok címei

		
	\end{description}
\item Metódusok\\
	\begin{description}
		\item[ArrayList<Enemy> hasEnemy()] visszaadja a rajta lévő ellenségek listáját
		\item[bool isPath()] igaz értékkel tér vissza
		\item[void deleteIPathPlaceable(IPathPlaceable ipath)] kitörli a tárolójából a paraméterként kapott referenciával megegyező tárolt referenciát
		\item[void registerIPathPlaceable(IPathPlaceable ipath)] beregisztrálja a paraméterként kapott objektumot, mint saját magán tartózkodó ellenség
		\item[bool hasEnemy()] megmutatja, hogy van-e a cellán ellenség
		\item[void registerEnemy(Enemy e)] a kapott ellenséget beteszi az enemies-be
		\item[void registerObstacel(Obstacle o)] a o kapott akadály lesz az obstacle
		\item[ArrayList<Enemy> getEnemies()] visszatér az enemies-el
		\item[Path getNext()] paths-ből ad vissza egy elemet
		
		
	\end{description}
\end{itemize}

\subsection{Towerhez tartozó krsitályok: Range, Speed, Damage, EnemyType}
\begin{itemize}
\item Felelősség\\
Ld objektum katalógus
\item Ősosztályok\\
Object $\rightarrow$ Gem
\item Interfészek\\
ITGem
\item Attribútumok\\
	\begin{description}
		\item[int range/ speed/ damage/ eType] 
		
	\end{description}
\item Metódusok\\
	\begin{description}
		\item[Konstruktorok]
		
	\end{description}
\end{itemize}
\subsection{Obstaclehez tartozó kristályok: Intensity, Repair}
\begin{itemize}
\item Felelősség\\
Ld objektum katalógus
\item Ősosztályok\\
Object $\rightarrow$ Gem
\item Interfészek\\
IOGem

\item Attribútumok\\
	\begin{description}
		\item[int intensity] 
		
	\end{description}
\item Metódusok\\
	\begin{description}
		\item[Konstruktorok]
		
	\end{description}
\end{itemize}
\subsection{IOGem}
\begin{itemize}
\item Felelősség\\
Akadályra helyezhető kristályok interfésze.

\item Metódusok\\
	\begin{description}
		\item[void upgradeObstacle(Obstacle o)] a kapott akadályt fejleszti.
		
	\end{description}
\end{itemize}
\subsection{ITGem}
\begin{itemize}
\item Felelősség\\
Toronyra illeszthető kristályok interfésze.

\item Metódusok\\
	\begin{description}
		\item[void upgradeTower(Tower t)] fejleszti a kapott t tornyot magával 
\item[int getValue()] visszaad egy, a torony árával képzett értéket, a torony eladásakor jóváírandó mana érték kiszámításához 

		
	\end{description}
\end{itemize}
\subsection{IFieldPlaceable}
\begin{itemize}
\item Felelősség\\
Interfész a mezőre helyezhető osztályok számára. Azonosítja azokat az objektumokat, amelyeket csak a mező típusú pályaelem tartalmazhat.

\item Metódusok\\
	\begin{description}
		\item[void registerField(Field field)] a mezőre helyezhető objektumnak megadja  paraméterben annak a mezőnek a referenciáját, amelyikre helyezve lesz.
\item[void sell()] a mezőre helyezhető objektum eladása, annak megfelelő manát ad a játékosnak amennyit az objektum ér, majd a mező törli magáról az objektumot.

		
	\end{description}
\end{itemize}
\subsection{IPathPlaceable}
\begin{itemize}
\item Felelősség\\
Interfész az útra helyezhető osztályok számára. Azonosítja azokat az objektumokat, amelyeket csak az út típusú pályaelem tartalmazhat.

\item Metódusok\\
	\begin{description}
		\item[void eliminate(Path p)] az útra helyezhető objektum eltávolítása az útról.
\item[void registerPath(Path p)] az útra helyezhető objektumnak megadja paraméterben annak az útnak a referenciáját, amelyikre helyezve lesz.



		
	\end{description}
\end{itemize}


\section{Szekvencia diagramok}
%\comment{Inicializálásra, use-case-ekre, belső működésre. Konzisztens kell legyen az előző alfejezettel. Minden metódus, ami ott szerepel, fel kell tűnjön valamelyik szekvenciában. Minden metódusnak, ami szekvenciában szerepel, szereplnie kell a valamelyik osztálydiagramon.}
\subsection{Akadály elhelyezése}
\begin{figure}[H]
\begin{center}
\includegraphics[width=17cm]{chapters/chapter04/images/Akadaly_elhelyezese.jpg}
\caption{Akadály elhelyezése szekvenciadiagram}
\label{fig:Akadály_elhelyezése}
\end{center}
\end{figure}

\subsection{Ellenfél mozgása}
\begin{figure}[H]
\begin{center}
\includegraphics[width=17cm]{chapters/chapter04/images/Ellenfel_mozgasa.jpg}
\caption{Ellenfél mozgása szekvenciadiagram}
\label{fig:Ellenfél_mozgása}
\end{center}
\end{figure}

\subsection{Torony eladása}
\begin{figure}[H]
\begin{center}
\includegraphics[width=17cm]{chapters/chapter04/images/Torony_eladasa.jpg}
\caption{Torony eladása szekvenciadiagram}
\label{fig:Torony_eladása}
\end{center}
\end{figure}

\subsection{Torony elhelyezése}
\begin{figure}[H]
\begin{center}
\includegraphics[width=17cm]{chapters/chapter04/images/Torony_elhelyezese.jpg}
\caption{Torony elhelyezése szekvenciadiagram}
\label{fig:Torony_elhelyezése}
\end{center}
\end{figure}

\subsection{Torony fejlesztése}
\begin{figure}[H]
\begin{center}
\includegraphics[width=17cm]{chapters/chapter04/images/Torony_fejlesztese.jpg}
\caption{Torony fejlesztése szekvenciadiagram}
\label{fig:Torony_fejlesztése}
\end{center}
\end{figure}

\subsection{Torony tüzelése}
\begin{figure}[H]
\begin{center}
\includegraphics[width=17cm]{chapters/chapter04/images/Torony_tuzelese.jpg}
\caption{Torony tüzelése szekvenciadiagram}
\label{fig:Torony_tüzelése}
\end{center}
\end{figure}

\subsection{Inicializálás}
\begin{figure}[H]
\begin{center}
\includegraphics[width=17cm]{chapters/chapter04/images/Inicializalas.jpg}
\caption{Inicializálás szekvenciadiagram}
\label{fig:Inicializálás}
\end{center}
\end{figure}

\subsection{Játék menete}
\begin{figure}[H]
\begin{center}
\includegraphics[width=17cm]{chapters/chapter04/images/Jatek_menete.jpg}
\caption{Játék menete szekvenciadiagram}
\label{fig:Játék_menete}
\end{center}
\end{figure}


\section{State-chartok}
%\comment{Csak azokhoz az osztályokhoz, ahol van értelme. Egyetlen állapotból álló state-chartok ne szerepeljenek. A játék működését bemutató state-chart-ot készíteni tilos.}

