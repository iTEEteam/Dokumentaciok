% Szglab4
% ===========================================================================
%
\chapter{Prototípus koncepciója}
\setlength{\parindent}{0cm}


\thispagestyle{fancy}
\setcounter{section}{-1}
\section{Specifikáció változás}
\subsection{Módosítás}
A tornyokra időnként köd ereszkedik, aminek következtében a látás erősen lecsökken. Ez hatással van a lövésre. A játékosok által járható útvonalon lehetnek elágazások és becsatlakozások. Az elágazásokon az egyes játékosok véletlenszerűen mennek a különböző irányokba. A tornyokban elvétve lehetnek olyan lövedékek, amelyek az eltalált játékost kettőbe vágják. A két játékos egymástól függetlenül él tovább, csökkentett életerővel.

A fentiek fényében a következő módosításokat hajtottuk végre:
\begin{itemize}


\item A köd torony hatósugarán változtat, így amíg a köd tart, addig a torony rövidebbre lát el, mint addig. Köd alatt is lehet fejleszteni a tornyot, ekkor ugyanúgy érvényre jut, de a köd által módosított hatókör fog nőni, majd a köd elmúltával olyan érték áll be, mintha nem lett volna köd és fejlesztették volna a tornyot. Köd véletlenszerűen keletkezik és minden tornyot érint. A köd időtartamát  Game osztály hazeTime attribútuma szabja meg. Ez mindig ugyanannyi idő, az attribútum csak arról gondoskodik, hogy ha letelt, akkor visszaálljon minden (számol). A torony interfészébe bekerültek a haze() és clearUp() metódusok, előbbi ködbe borít, utóbbi kitisztít. A torony kapott egy modRange attribútumot, aminek segítségével kényelmesen kezelhető a ködbe borulás.
\item Az elágazások és becsatlakozások tekintetében nem változtattunk, a modellen, mert az eddigiek alapján ezt már eddig is tudta. Vagyis amikor az Enemy továbblép, meghívódik a getNext() metódus, ami a következő Path címével tér vissza annak a Path-nak a nextPaths listájából, amelyikről lépni akar. Ha a listában több Path van, az pont az elágazásnak felel meg, ekkor véletlenszerűen adunk egy címet.
\item Amikor egy Enemy-n meghívjuk a hurt metódust, átadunk egy Bulletet, amin az adott típusú Enemy meghívja a rá jellemző getDamage metódust, hogy megtudja a sebzést. Időnként 0-t kap majd, aminek hatására a meghívódik az Enemy-ben felvett cut() metódus, ami létrehoz egy ugyanilyen típusú Enemy-t (felüldefiniálás révén), aminek feleakkora életerőt ad, mint amennyi neki van, valamint a sajátját is felezi, továbbá minden egyéb attribútum értékét átmásolja az „ikertestvérébe”. Ezen felül, mivel az „ikertestvér”nek is be kell kerülnie a pályán lévő enemy-k közé, hogy vezérelhessük, az IGame interfészbe tettünk egy addEnemy fv-t, hogy az Enemyi betehesse az új Enemy-t.
\end{itemize}
\subsection{Módosított osztálydiagram}

\begin{figure}[H]
\begin{center}
\includegraphics[width=17cm]{chapters/chapter07/images/MainModSpec.jpg}
\caption{Módosított osztálydiagram}
\label{fig:Modositott_Osztálydiagram}
\end{center}
\end{figure}

\subsection{Módosított szekvenciadiagramok}
A módosított diagramok alapjai az analízis modell szekvenciadiagramjai.

\subsubsection{Játék menete}
\begin{figure}[H]
\begin{center}
\includegraphics[width=13cm]{chapters/chapter07/images/sd_Jatek_menete.jpg}
\caption{Játék menete szekvenciadiagram}
\label{fig:Játék_menete2}
\end{center}
\end{figure}

\subsubsection{Torony kettévág egy ellenséget}
\begin{figure}[H]
\begin{center}
\includegraphics[width=17cm]{chapters/chapter07/images/sd_torony_kettevag_egy_ellenseget.jpg}
\caption{Torony kettévág egy ellenséget szekvenciadiagram}
\label{fig:Torony_kettevag_egy_ellenseget}
\end{center}
\end{figure}



\section{Prototípus interface-definíciója}
%\comment{Definiálni kell a teszteket leíró nyelvet. Külön figyelmet kell fordítani arra, hogy ha a rendszer véletlen elemeket is tartalmaz, akkor a véletlenszerűség ki-bekapcsolható legyen, és a program determinisztikusan is tesztelhető legyen.}




\subsection{Az interfész általános leírása}
%\comment{A protó (karakteres) input és output felületeit úgy kell kialakítani, hogy az input fájlból is vehető legyen illetőleg az output fájlba menthető legyen, vagyis kommunikációra csak a szabványos be- és kimenet használható.}
Az interfész teljesen karakteres alapú, a szabványos ki- és bemenetet használja. Indítás után a 7.1.2 pontban leírt parancsokat lehet megadni, és a 7.1.3 pontban leírt módon kapjuk a kimenetet. 

Lehetséges a ki- és bemenet átirányítása is, így például előre összeállított teszteseteket elmenthetünk fájlban, és átirányítással megadhatjuk a programnak, továbbá a kimenetét is fájlba irányíthatjuk a könnyebb elemzés érdekében. 

Az átirányítás a hagyományos módon történik, például a 
\begin{verbatim}
java „osztálynév” < bemenet.txt > kimenet.txt
\end{verbatim}
parancs hatására a program megkapja soronként a bemenet.txt tartalmát, és a kimenet.txt fájlba írja a kimenetét.

\subsection{Bemeneti nyelv}
%\comment{Definiálni kell a teszteket leíró nyelvet. Külön figyelmet kell fordítani arra, hogy ha a rendszer véletlen elemeket is tartalmaz, akkor a véletlenszerűség ki-bekapcsolható legyen, és a program determinisztikusan is futtatható legyen. A szálkezelést is tesztelhető, irányítható módon kell megoldani.}
A bemeneti nyelv azoknak az utasításoknak a halmazát jelenti, amikkel a Prototípust vezérelni, és ezáltal tesztelni lehet.

Egy sorban 1 utasítás megadására van lehetőség. Az üres sorokat a parser figyelmen kívül hagyja. Minden parancs esetén a paramétereket szóközzel elválasztva kell felsorolni. Az utasítás végét nem kell pontosvesszővel lezárni, mert minden utasítás új sorba kerül.

\begin{itemize}
\item loadmap
	\begin{description}
	\item[Leírás:] Betölti a pályát.
	\item[Opciók:] \#1 $\rightarrow$ filename
	\item[Példa:] loadmap tesztpalya.txt
	\end{description}

\item update
	\begin{description}
	\item[Leírás:] 	Az idő múlását vezérli. Egy egész szám a paramétere, hogy hány tick fusson le.
	\item[Opciók:] \#1 $\rightarrow$ [0-9]+
	\item[Példa:] update 10 
	\end{description}

\item draw
	\begin{description}
	\item[Leírás:] Kirajzolja a pályát, és a fontosabb adatokat: 
	\begin{itemize}
	\item mennyi manája van a játékosnak
	\item egy torony mennyibe kerül
	\item egy akadály mennyibe kerül
	\item egy kristály mennyibe kerül
	\end{itemize}
	\item[Opciók:] --
	\item[Példa:] draw
	\end{description}

\item buy 
	\begin{description}
	\item[Leírás:] 	Objektumok (torony, akadály) elhelyezése a pályán, és fejlesztésük Gemekkel.
	\item[Opciók:] 
	
	\#1 $\rightarrow$ [tower | obstacle | gem] 					(mit veszünk)
	
	\#2 $\rightarrow$ x-y 								(hova vesszük)
	
	\#3 $\rightarrow$ [speed | range | damage | enemy | inensity | repair]		(milyen gem)
	
	\#4 $\rightarrow$ [elf | hobbit | dwarf | human]					(milyen ellenségre)
	
3. paraméter csak Gem esetén létezik, míg 4. paraméter csak Enemy-Gem esetén.

	\item[Példa:] 
	\begin{itemize}
	\item buy tower 5-3
	\item buy obstacle 10-12
	\item buy gem 5-3 enemy hobbit
\end{itemize}		
	

	\end{description}

\item random
	\begin{description}
	\item[Leírás:] 	A véletlenszerűséget tudja ki és bekapcsolni.
	\item[Opciók:] \#1 $\rightarrow$ [true | false]
	\item[Példa:] random true
	\end{description}

\item exit
	\begin{description}
	\item[Leírás:] 	Kilép a programból.
	\item[Opciók:] --
	\item[Példa:] exit
	\end{description}


\end{itemize}

%\comment{Ha szükséges, meg kell adni a konfigurációs (pl. pályaképet megadó) fájlok nyelvtanát is.}

\subsubsection{A pálya formátuma}
A pályát karakterekkel rajzoljuk ki. 
Szabályok: pálya széle mindenhol "field", kivéve "entry point" és "exit point"
\begin{description}
\item[x]	field
\item[en]	entry point
\item[r]	right
\item[l]	left
\item[u]	up
\item[d]	down
\item[ru]	right, up
\item[rl] 	right, left
\item[rd]	right, down
\item[ul]	up, left
\item[ud]	up, down
\item[dr]	down, right
\item[rul]	right, up, left
\item[rud]	right, up, down
\item[rdl]	right, down, left
\item[dul]	down, up, left
\item[ex]	exit point	
\end{description}



\subsection{Kimeneti nyelv}
%\comment{Egyértelműen definiálni kell, hogy az egyes bemeneti parancsok végrehajtása után előálló állapot milyen formában jelenik meg a szabványos kimeneten.}

Egy bemeneti parancs után ha a programnak lényeges függvénye hívódik meg, kiírja azt.

Ha helyénvaló, az objektum katalógus beli nevét is kiírja.

Ha több függvényhíváson keresztül ír ki, minden függvény mélyén egy tabbal beljebb ír ki.
\\[12pt]
Bemenet: loadmap

Kimenet: nincs
\\[6pt]

Bemenet: update <n>

Lehetséges kimenetek: 
\begin{itemize}
\item update –- minden frissítés elején kiírja
\item makeEnemies –- ha a pályán kívül várakozó és bent lévő ellenségekről tárolt tömb is üres.
\item \emph{katalógus név}  created –- egyes ellenségek konstruktorában kiírja a katalógus nevét
\item katalógus név move –- ha egy ellenség lép, ezt írja ki
\item katalógus név crossroad –- ha ellenségelágazáshoz ért
\item katalógus név shoot katalógus név target –- ha egy torony ellenségre lő, ezt írja ki
\item haze –- amikor köd ereszkedik a tornyokra  
\end{itemize}

Példa: 
\begin{lstlisting}
update 2
update
	makeEnemies
		Hobbit0 created
		Hobbit1 created
		Dwarf0 created
update
	Hobbit0 move
	Hobbit0 crossroad
\end{lstlisting}

Bemenet: draw

Kimenet: 
\begin{itemize}
\item Fentebb definiált módon, táblázatosan kirajzolja a pályát. A pálya egy celláját egy számpár határozza meg, az oszlop és sor koordinátája.
\item Your mana: n – a játékos manája
\item Tower price: n –  egy torony ára
\item Obstacle price: n – egy akadály ára
\item Gem price: n – egy kristály ára
\end{itemize} 

Példa: 
\begin{lstlisting}
draw
	0	1	2	3	4	5
0	x	x	x	x	x	x
1	en	r	rd	r	d	x
2	x	x	d	x	d	x
3	x	x	d	x	d	x
4	x	x	r	r	r	ex
5	x	x	x	x	x	x
Your mana: 100
Tower price: 20
Obstacle price: 10
Gem price: 50
\end{lstlisting}

Bemenet: buy <tpye><coord><gem><enemy>

Lehetséges kimenetek:
\begin{itemize}
\item katalógus név created – a létrejövő objektum kiírja a katalógusban tárolt nevét
\item cell occupied – ha a kapott koordinátán nem lehet elhelyezni az objektumot
\end{itemize}

Példa1:
\begin{lstlisting}
buy tower 5-3
Tower0 created
\end{lstlisting}

Példa2: 
\begin{lstlisting}
buy gem 5-3 enemy hobbit
HobbitGem0 created
\end{lstlisting}


Bemenet: random

Kimenet: nincs
\\[6pt]

Bemenet: exit

Kimenet: nincs


\section{Összes részletes use-case}
%\comment{A use-case-eknek a részletezettsége feleljen meg a kezelői felületnek, azaz a felület elemeire kell hivatkozniuk.
%Alábbi táblázat minden use-case-hez külön-külön.}

%\begin{figure}[h]
%\begin{center}
%%\includegraphics[width=17cm]{chapters/chapter07/example.pdf}
%\caption{x}
%\label{fig:ProtoUseCase}
%\end{center}
%\end{figure}

\usecase
{Játék inicializálása}
{A játék indulása után, inicializálja a változókat.}
{Application, Game}
{}

\usecase
{Ellenség lépése}
{Egy ellenség az egyik celláról a másikra lép.}
{Game}
{Az enemy elkéri a cellájától a következő cellát, törölteti magát az aktuális celláról, és beregisztráltatja magát a következőre.}

\usecase
{Ellenség meghal}
{Egy ellenség meghal, tehát teljesen eltűnik.}
{Game}
{Az enemy törölteti magát az összes enemy listájából (Game), és az aktuális cellájából is.}

\usecase
{Ellenség bejut a végzet hegyéhez}
{Egy ellenség bejut a végzet hegyéhez. A játék vége.}
{Game}
{Annyiban tér el a sima mozgástól, hogy a Game ellenőrzi, hogy Szarumán varázslata elfogyott-e már, és ha igen, akkor véget ér a játék.}

\usecase
{Ellenség elágazáshoz ér}
{Egy ellenség egy út-celláról legalább 2 másik cellára léphet tovább.}
{Game}
{Az ellenség mozgásának egy speciális esete. Amikor az enemy elkéri a Path-tól a következő cellát, akkor beállítástól függően vagy determinisztikusan a listában szereplő első cellát kapja meg, vagy veletlenszerűen a listából egyet.}

\usecase
{Torony megvétele}
{Üres mezőre tornyot vesz a felhasználó.}
{Controller, Tower}
{Először a Controller setField metódusával kiválaszt a felhasználó egy mezőt amire lerak egy tornyot.}

\usecase
{Toronyra kristály vétele}
{Torony fejlesztése kristállyal.}
{Controller, Tower}
{A felhasználó kiválaszt egy mezőt, és ha van rajta torony fejleszti egy kiválasztott kristállyal.}

\usecase
{Torony eladása}
{Torony eladása.}
{Controller, Tower}
{A felhasználó kiválaszt egy mezőt, és a rajta lévő tornyot eladja.}

\usecase
{Toronyra köd ereszkedik}
{Toronyra köd ereszkedik, ami a hatósugarát lecsökkenti.}
{Tower, Game}
{A Game kiválaszt egy tornyot, amire ködöt rak. Lecsökken a hatósugara a toronynak, ezért frissíti az út cellák listáját. (setPath)}

\usecase
{Torony lő}
{Egy, a torony a hatósugarában lévő ellenségre tüzel az ellenség.}
{Tower, Enemy}
{A torony lekér a hatósugarában lévő mezők közül egy ellenséget, amire kilövi a bullet-jét.}

\usecase
{Akadály megvétele}
{Egy üres útra akadályt vesz a játékos.}
{Controller, Obstacle}
{Először a Controller setPath metódusával kiválaszt a felhasználó egy mezőt amire lerak egy tornyot.}

\usecase
{Akadály lassít}
{A lerakott akadály lassítja a rajta áthaladó ellenséget.}
{Path}
{Amikor egy ellenség egy akadállyal terhelt cellára lép, akkor a Path beállítja az enemy modSpeed-jét.}

\usecase
{Akadály elromlik}
{Akadály elromlik}
{Path}
{A path törli az akadályt.}

\usecase
{Akadály eladás}
{A játékos elad egy akadályt.}
{Controller}
{A felhasználó kiválaszt egy utat, és a rajta lévő akadályt eladja.}


\section{Tesztelési terv}
%\comment{A tesztelési tervben definiálni kell, hogy a be- és kimeneti fájlok egybevetésével miként végezhető el a program tesztelése. Meg kell adni teszt forgatókönyveket. Az egyes teszteket elég informálisan, szabad szövegként leírni. Teszt-esetenként egy-öt mondatban. Minden teszthez meg kell adni, hogy mi a célja, a proto mely funkcionalitását, osztályait stb. teszteli. Az alábbi táblázat minden teszt-esethez külön-külön elkészítendő.}

\teszteset
{Ellenség hullám indítása}
{A Game létrehoz 3 ellenséget, a játékos vesz 2 tornyot, és mindkettőt fejleszti 1-1 kristállyal. Ez után utukra indítja az ellenségeket. Végén elad a játékos egy tornyot.}
{Ez a teszteset az ellenség hullám létrehozását, tornyok vételét, fejlesztését, tüzelését és eladását teszteli. 
Ebben a tesztesetben fogunk látni ellenség halálát is. A játékos nyer.
}

\teszteset
{Akadály és ellenség bejut a végzet hegyéhez}
{A pályán elhelyez a játékos egy akadályt, amit fejleszt egy intenzitás növelő kristállyal, majd bejut az ellenség a végzet hegyéhez. Amikor az utolsó ellenség is áthaladt rajta elhasználódik.}
{A teszteset akadály vételt, fejlesztést, működést és elhasználódást, teszteli, illetve a végzet hegyéhez való bejutást. A játékos veszít.}

\teszteset
{Köd és ellenség kettészelés}
{A pályára feltesz a Game 3 ellenséget, a játékos vesz 3 tornyot, amikor beér az első ellenség a torony hatósugarába, köd száll a tornyokra, és kettőbe vágó lövedékeket lőnek onnantól. Miközben végigérnek az ellenségek a pályán, meghalnak.}
{Ez a teszteset a tornyokra ereszkedő ködöt teszteli és a kettészelő lövedékeket. }

\teszteset
{Ellenség elágazáshoz ér}
{A pályára feltesz a Game 3 ellenséget.
A pálya kialakításából adódóan 1 lépés után elágazáshoz érnek. 
}
{A teszteset azt teszteli elágazásoknál merre mennek tovább az ellenségek.
Ha a véletlenszerűség be van kapcsolva a programban, akkor egy véletlen utat választ, ha nem akkor a 0. elemét adja vissza az utakat tároló tömbnek.
}


\section{Tesztelést támogató segéd- és fordítóprogramok specifikálása}
%\comment{Specifikálni kell a tesztelést támogató segédprogramokat.}
A prototípustól adott bemenet esetén elvárt, illetve a kapott eredmények összehasonlítását saját programmal végezzük. Az összehasonlító program parancssorból futtatható, két paramétert vár: a várt és a kapott eredményeket tartalmazó fájlok neveit. A program lefutása után megmondja, hogy a két fájl megegyezik-e, és ha nem, akkor kiírja az eltérő sorok sorszámát.

A program futtatása a következő parancs kiadásával történik, a lefordított .class file mellett:
\begin{verbatim}
java TxtComparer <elso_file_neve> <masodik_file_neve>
\end{verbatim}

