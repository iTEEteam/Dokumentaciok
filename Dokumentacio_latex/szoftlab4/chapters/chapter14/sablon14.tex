% Szglab4
% ===========================================================================
%
\chapter{Összefoglalás}

\thispagestyle{fancy}

\section{Projekt összegzés}
%\comment{A projekt tapasztalatait összegző részben a csapatoknak a projektről kialakult véleményét várjuk. A megválaszolandók köre az alábbi. }

\begin{munka}
\munkaido{Elekes}{80}
\munkaido{Fuksz}{68}
\munkaido{Nagy}{40}
\munkaido{Rédey}{74}
\munkaido{Seres}{91}
\osszesmunkaido{353}
\end{munka}

\begin{forrassor}
\munkaido{Szkeleton}{2091}
\munkaido{Protó}{2576}
\munkaido{Grafikus}{3761}
\end{forrassor}

\begin{itemize}
\item Mit tanultak a projektből konkrétan és általában? %\newline
\begin{itemize}
\item A project lehetőséget nyújtott kipróbálni magunkat csapatmunkában és a RUP processz szerinti szoftver fejlesztést. Ez új fajta kihívásokat jelentett, csapaton belüli kommunikációt, közös munkát segítő szoftverek használatát kellett megtanulni. 
\item Git verziókezelő megtanulása volt első feladatunk, de kezdeti nehézségek után rendkívül hasznos eszköznek bizonyult.
\item Fontos volt, hogy feladatok szétosztásánál figyeljünk arra mik azok amik egymásra épülnek, és egyenletesen legyenek szétosztva.
\end{itemize}
\item Mi volt a legnehezebb és a legkönnyebb? %\newline
\begin{itemize}
\item Na ez egy nehéz kérdés. Egyes pontok megértése a különböző dokumentumokban, végiggondolni, hogy pontosan mit is kellene oda írni, ez sok fejtörést okozott néha. Maga a modell és a kód szinkronban tartása, a különböző változatok böngészése sem volt egyszerű. 
\item Illetve a tesztágy elkészítése a szkeleton és prototípus esetén elég kemény volt. 
\item Ugyanakkor a már kész modellt lekódolni már nem is volt annyira nagy feladat.
\end{itemize}

\item Összhangban állt-e az idő és a pontszám az elvégzendő feladatokkal?% \newline
\begin{itemize}
\item Az egyes feladatokra járó pontszám, és idő reálisnak mondható.
\end{itemize}
\item Ha nem, akkor hol okozott ez nehézséget? \newline
\item Milyen változtatási javaslatuk van? %\newline
\begin{itemize}
\item A leadási határidő hétfő helyett lehetne kedden, így elkerülhető lenne a vasárnap esti rohanás, ami vidéki hallgatóknak a felutazás idejébe esik, így az utolsó simításokban kevésbé tudnak részt venni.
\item Az egy főre jutó vég pontszám egyértelműbb leírása a tárgyhonlapon jó lenne.
\end{itemize}

\item Milyen feladatot ajánlanának a projektre?% \newline
\begin{itemize}
\item Feladatnak úton/folyón békával átkelős játékot ajánlanánk, a folyón különböző sebességgel úszó farönkökön kelljen átmenni, ha kiúszunk rajta a pályáról az életet vesz le.
\item Úton különböző sebességű autók között kell úgy átjutni, hogy ne üssenek el. Csavarként, esetleg időnként jöhet egy gólya, aki random helyre visszarakja a főhősünket.
\end{itemize}
\end{itemize}

%\comment{Szívesen fogadunk minden egyéb kritikát és javaslatot.}