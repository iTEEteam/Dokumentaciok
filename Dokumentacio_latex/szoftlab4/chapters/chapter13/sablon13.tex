% Szglab4
% ===========================================================================
%
\chapter{Grafikus felület specifikációja}

\thispagestyle{fancy}

\section{Fordítási és futtatási útmutató}
%\comment{A feltöltött program fordításával és futtatásával kapcsolatos útmutatás. Ennek tartalmaznia kell leltárszerűen az egyes fájlok pontos nevét, méretét byte-ban, keletkezési idejét, valamint azt, hogy a fájlban mi került megvalósításra.}

\subsection{Fájllista}

\begin{fajllista}

\fajl
{graph/Application.java}
{}
{2014.03.11.}
{Névvel megegyező osztály.}

\fajl
{graph/Bullet.java}
{}
{2014.03.11.}
{Névvel megegyező osztály.}

\fajl
{graph/Cell.java}
{}
{2014.03.11.}
{Névvel megegyező osztály.}

\fajl
{graph/Controller.java}
{}
{2014.03.11.}
{Névvel megegyező osztály.}

\fajl
{graph/DamageGem.java}
{}
{2014.03.11.}
{Névvel megegyező osztály.}

\fajl
{graph/Dwarf.java}
{}
{2014.03.11.}
{Névvel megegyező osztály.}

\fajl
{graph/Elf.java}
{}
{2014.03.11.}
{Névvel megegyező osztály.}

\fajl
{graph/Enemy.java}
{}
{2014.03.11.}
{Névvel megegyező osztály.}

\fajl
{graph/EnemyTypeGem.java}
{}
{2014.03.11.}
{Névvel megegyező osztály.}

\fajl
{graph/Field.java}
{}
{2014.03.11.}
{Névvel megegyező osztály.}

\fajl
{graph/Game.java}
{}
{2014.03.11.}
{Névvel megegyező osztály.}

\fajl
{graph/GCell.java}
{}
{2014.03.11.}
{Névvel megegyező osztály.}

\fajl
{graph/GController.java}
{}
{2014.03.11.}
{Névvel megegyező osztály.}

\fajl
{graph/GDwarf.java}
{}
{2014.03.11.}
{Névvel megegyező osztály.}

\fajl
{graph/GElf.java}
{}
{2014.03.11.}
{Névvel megegyező osztály.}

\fajl
{graph/Gem.java}
{}
{2014.03.11.}
{Névvel megegyező osztály.}

\fajl
{graph/GEnemy.java}
{}
{2014.03.11.}
{Névvel megegyező osztály.}

\fajl
{graph/GField.java}
{}
{2014.03.11.}
{Névvel megegyező osztály.}

\fajl
{graph/GGame.java}
{}
{2014.03.11.}
{Névvel megegyező osztály.}

\fajl
{graph/GHobbit.java}
{}
{2014.03.11.}
{Névvel megegyező osztály.}

\fajl
{graph/GHuman.java}
{}
{2014.03.11.}
{Névvel megegyező osztály.}

\fajl
{graph/GPath.java}
{}
{2014.03.11.}
{Névvel megegyező osztály.}

\fajl
{graph/Graphic.java}
{}
{2014.03.11.}
{Névvel megegyező osztály.}

\fajl
{graph/Hobbit.java}
{}
{2014.03.11.}
{Névvel megegyező osztály.}

\fajl
{graph/Human.java}
{}
{2014.03.11.}
{Névvel megegyező osztály.}

\fajl
{graph/IFieldPlaceable.java}
{}
{2014.03.11.}
{Névvel megegyező osztály.}

\fajl
{graph/IGame.java}
{}
{2014.03.11.}
{Névvel megegyező osztály.}

\fajl
{graph/IntensityGem.java}
{}
{2014.03.11.}
{Névvel megegyező osztály.}

\fajl
{graph/IObstacle.java}
{}
{2014.03.11.}
{Névvel megegyező osztály.}

\fajl
{graph/IOGem.java}
{}
{2014.03.11.}
{Névvel megegyező osztály.}

\fajl
{graph/IPathPlaceable.java}
{}
{2014.03.11.}
{Névvel megegyező osztály.}

\fajl
{graph/ITGem.java}
{}
{2014.03.11.}
{Névvel megegyező osztály.}

\fajl
{graph/ITower.java}
{}
{2014.03.11.}
{Névvel megegyező osztály.}

\fajl
{graph/IView.java}
{}
{2014.03.11.}
{Névvel megegyező osztály.}

\fajl
{graph/Map.java}
{}
{2014.03.11.}
{Névvel megegyező osztály.}

\fajl
{graph/Obstacle.java}
{}
{2014.03.11.}
{Névvel megegyező osztály.}

\fajl
{graph/Path.java}
{}
{2014.03.11.}
{Névvel megegyező osztály.}

\fajl
{graph/RangeGem.java}
{}
{2014.03.11.}
{Névvel megegyező osztály.}

\fajl
{graph/RepairGem.java}
{}
{2014.03.11.}
{Névvel megegyező osztály.}

\fajl
{graph/ResourcesCache.java}
{}
{2014.03.11.}
{Névvel megegyező osztály.}

\fajl
{graph/SpeedGem.java}
{}
{2014.03.11.}
{Névvel megegyező osztály.}

\fajl
{graph/Tower.java}
{}
{2014.03.11.}
{Névvel megegyező osztály.}

\fajl
{gameImages/field.jpg}
{}
{2014.05.07.}
{}

\fajl
{gameImages/path.jpg}
{}
{2014.05.07.}
{}

\fajl
{gameImages/dwarf.png}
{}
{2014.05.07.}
{}

\fajl
{gameImages/elf.png}
{}
{2014.05.07.}
{}

\fajl
{gameImages/hobbit.png}
{}
{2014.05.07.}
{}

\fajl
{gameImages/human.png}
{}
{2014.05.07.}
{}

\fajl
{gameImages/tower1.png}
{}
{2014.05.07.}
{}

\fajl
{gameImages/tower2.png}
{}
{2014.05.07.}
{}

\fajl
{gameImages/tower3.png}
{}
{2014.05.07.}
{}

\fajl
{gameImages/tower4.png}
{}
{2014.05.07.}
{}

\fajl
{gameImages/tower5.png}
{}
{2014.05.07.}
{}










\fajl
{GraphicCompileAndRun.bat}
{1128}
{2014.05.06.}
{Program fordítása és futtatása}


\end{fajllista}

\lstset{escapeinside=`', xleftmargin=10pt, frame=single, basicstyle=\ttfamily\footnotesize, language=sh}

\subsection{Fordítás}
%\comment{A fenti listában szereplő forrásfájlokból milyen műveletekkel lehet a bináris, futtatható kódot előállítani. Az előállításhoz csak a 2. Követelmények c. dokumentumban leírt környezetet szabad előírni.}


A fordítás a ''javac'' programmal történik, ezért olyan környezetben lehetséges csak, ahol ez a program elérhető. 

A tesztelés megkönnyítése végett a fordítás és futtatás automatizálására egy batch fájl áll rendelkezésre. Ennek a használata a 13.1.3 pontban van részletezve. 

A fordítás parancssorból is lehetséges a következő, a project gyökerében kiadott paranccsal:

\begin{lstlisting}
if exist bin rmdir bin /s /q
mkdir bin
javac -d ./bin/ graph/*.java
\end{lstlisting}

Ez a parancs lefordítja a programot, és a lefordított állományokat a bin mappába helyezi.  



\subsection{Futtatás}
%\comment{A futtatható kód elindításával kapcsolatos teendők leírása. Az indításhoz csak a 2. Követelmények c. dokumentumban leírt környezetet szabad előírni.}

A futtatás a ''java'' programmal történik, ezért olyan környezetben lehetséges csak, ahol ez a program elérhető. 

Az adott batch fájlok segítségével lehetséges lefordítani és futtatni a programot: ehhez a GraphicCompileAndRun.bat-ot futtassuk, ami a fordítás után azonnal el is indítja a programot.

Ha a futtatást parancssorból akarjuk végezni, lehetőség van arra is. Amennyiben a 13.1.2 pontban leírt módon végeztük a fordítást, a futtatás a következő, a project gyökerében kiadott paranccsal végezhető:

\begin{lstlisting}
java -classpath .\bin\ graph.Application
\end{lstlisting}
Ez a parancs elindítja a programot.

\section{Értékelés}
%\comment{A projekt kezdete óta az értékelésig eltelt időben tagokra bontva, százalékban.}

%\begin{ertekelesplusz}
%\tag{Elekes} % Tag neve
%{} %óra
%{}        % Munka szazalekban
%\tag{Fuksz}
%{}
%{}
%\tag{Nagy}
%{}
%{}
%\tag{Rédey}
%{}
%{}
%\tag{Seres}
%{}
%{}

